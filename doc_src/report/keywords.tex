%%% Local Variables: 
%%% mode: latex
%%% TeX-master: "main"
%%% End: 

\chapter{A glimpse of \moca}
\label{chap:keywords}

This chapter covers the equational specifications and the semantics of
the keywords used by \moca to extend the traditional \ocaml syntax.

If you want to see some coding examples using \moca, you can also
have a look at chapter \ref{chap:examples} and come back here to get all
the details.

\section{BNF of \moca constructs}
\label{sec:bnf}

We will give in this section the means of writing a syntactically correct \mlm file
by highlighting what has been added to \ocaml (see Fig. \ref{fig:bnf}).

Non-terminals undefined in this BNF come from \ocaml ({\sf
  http://caml.inria.fr/pub/docs/manual-ocaml/language.html}). The
complete BNF (\moca plus \ocaml) can also be found in the {\sf
  parser.mly} file under the subdirectory {\sf
  compiler/ocaml\_src/parsing/} of the directory where you extracted moca.  

\begin{figure}[!ht]
  \centering
\begin{verbatim}
<constr-decl> ::= <constr-name> [ <annotation> ]
| <constr-name> of <typexpr> [ <annotation> ]
 
<annotation> ::= begin
               { | <relation>
                 | <completion_hint>
                 | <structure_item>
               }+
               end

<side> ::= left | right

<invopp> ::= inverse | opposite

<rpo_status> ::= lexicographic | multiset

<completion_hint> ::=
| completion precedence <int>
| completion status <rpo_status> 

<compare> ::= "(" val_longident ")"

<relation> ::=
| commutative [(<compare>)]
| associative
| involutive
| idempotent [<side>]
| neutral [<side>] "(" <constr-name> ")"
| nilpotent "(" <constr-name> ")"
| invopp [side] "(" <constr-name> ["," <constr-name>] ")"
| distributive [<invopp>] [<side>] "(" constr-name ["," <constr_name>]  ")"
| absorbent [<side>] "(" <constr-name> ")"
| absorbing [<side>] "(" <constr-name> ")"
| rule <pattern> -> <expression>
\end{verbatim}
  
  \caption{\moca BNF}
  \label{fig:bnf}
\end{figure}

After reading the BNF, one could now be able to construct
syntactically correct \moca 
programs but would not know what behavior to expect from the possible
keywords. Let us now turn to the specification of their (equational)
semantics. 

The {\sf structure\_item} part
allows to embed \ocaml code in the definition of
a relational data type. This is used in particular to define
comparison functions, whose identifier can then be used as argument to
the commutative keyword (\mysec{sec:commutative}).

\section{Keywords semantics}
\label{sec:kwd_specs}

The definition of the semantics of keywords is the first important
step towards having a correct code generation. Otherwise, how would we
know how to measure the correction of the code? We will describe here
an almost formal specification of the expected impact of \moca
keywords on the behavior of the corresponding programs.


\moca classifies generators (\ocaml constructors with relations)
according to their arity. We distinguish five categories of generators:
\begin{description}
\item[Zero-ary generators] are constant \ocaml constructors;
\item[Unary generators] have only one argument;
\item[Binary generators] have two;
\item[N-ary generators] have more than two but have a constant single arity;
\item[Listary generators] represent generators of variable arity and
  their only argument is a list. 
\end{description}


In all subsections below, we consider that the keyword is applied to a
constructor  \C. Also we use the following conventions:
\begin{itemize}
\item parentheses and commas $C(x, y)$ will delimit arguments of
  non-listary generators as in usual mathematical textbooks.
\item square brackets and semi-colons $C[x; y; z]$ will delimit arguments of
  listary generators as in \ocaml lists.
\end{itemize}

\subsection{Absorbent}
\label{sec:absorbent}

Examples of absorbent elements:
\begin{itemize}
\item $0$ for multiplication 
\item $\bot$ (aka {\sf false} in \ocaml) for the boolean $\land$ connector (aka \&\&)
\end{itemize}

Any absorbent element {\em must} be zero-ary. Otherwise, it is an error.

\begin{description}
\item[For zero-ary generators] \na.
\item[For unary generators]~
  \begin{itemize}
  \item \ift{absorbent (both, A)}{C(A) \rw A}
  \end{itemize}

In this case, absorbent (left, A) and absorbent (right, A) are errors.

\item[For binary generators]~
  \begin{itemize}
  \item \ift{absorbent (left, A)}{C(A, x) \rw A}
  \item \ift{absorbent (right, A)}{C(x, A) \rw A}
  \end{itemize}
  absorbent (both, a) is the combination of absorbent (left, a) and
  absorbent (right, a). 

\item[For listary generators]~
  \begin{itemize}
  \item
    \ift{absorbent (both, A)}
    {C & [b_1; ...; b_m; A; c1; ...; c_n] \rw A \\
      \text{In particular:} &  C [A] \rw A
    }

  \item \ift{absorbent (left, A)}
    { & C [b_1; ...; b_m; A; c1; ...; c_n] \rw C [b_1; ...; b_m; A] \\
      \text{In particular:} &  C [A] \rw A
    }
  \item \ift{absorbent (right, A)}
    {&
      C [b_1; ...; bm; A; c1; ...; cn] \rw C [A; c1; ...; cn] \\
      \text{In particular:} & C [A] \rw A
    }
  \end{itemize}

\item[For n-ary generators] \na

\end{description}

\note{ Absorbent (side, A) is (mostly) equivalent to
         Distributive (side, A, None, Dist\_Direct).}


\subsection{Absorbing}
\label{sec:absorbing}

The absorbing keyword is only applicable to binary generators. Other
uses are not defined\footnote{It is not clear what the intuition is in the
binary case is either.}.
\begin{description}

\item[For zero-ary generators] \na
\item[For unary generators] \na
\item[For binary generators] ~
  \begin{itemize}
  \item \ift{absorbing (left, D)}{C( D (x, y), y) \rw y}
  \item \ift{absorbing (right, D)}{C(x, D(x, y) ) \rw x}
  \end{itemize}



absorbing (both, D) is the conjunction of absorbing (left, D) and
absorbing (right, D). 
\item[For listary generators] \na
\item[For n-ary generators] \na

\end{description}

\subsection{Associative}
\label{sec:associative}

\note{As of today, side cannot be specified in \mlm files. The
  default is left.}

Examples of associative generators:
\begin{itemize}
\item multiplication on integers or reals 
 $ x * ( y * z) = ( x * y ) * z $
\end{itemize}

\begin{description}

\item[For zero-ary generators] \na
\item[For unary generators] \na
\item[For binary generators] ~
  \begin{itemize}
  \item \ift{associative (left)}{C( C (x, y), z) \rw C(x, C (y , z) ) }
  \item \ift{associative (right)}{C(x, C (y , z) ) \rw C( C (x, y), z) }
  \end{itemize}



associative (both) is equivalent to associative (left).
\item[For listary generators] 
\ift{associative (left)}
{C[x_1; ..., x_i; C [y_1; ...; y_j]; z_1; ...; z_k] \rw
 C [x_1; ..., x_i; y_1; ...; y_j; z_1; ...; z_k ] }

associative (both), associative (left) and associative (right)
are identical for listary generators.

\item[For n-ary generators] \na

\end{description}


\subsection{Commutative}
\label{sec:commutative}

The commutative keyword accepts an optional argument denoting the
identifier of a user-defined comparison function. The default is to
use \ocaml default total comparison function (ie commutative is
equivalent to commutative({\sf  Pervarsives.compare})).



\begin{description}

\item[For zero-ary generators] \na
\item[For unary generators] \na
\item[For binary generators]~
  \begin{itemize}
  \item
    \ift{commutative (cmp)}{
      C(x, y) \rw C (y, x) ~~\text{if {\sf cmp x y > 0}} \\
      \text{The arguments are sorted in increasing order with respect to cmp.}
    }

\item
\ift{commutative (cmp) and also has associative}{
C(x, C (y, z)) \rw C (y, C(x, z)) ~~\text{if {\sf cmp x y > 0}} \\
C(x, y) \rw C (y, x) ~~\text{if {\sf cmp x y > 0} and } y
\text{is not of the form } C (u, v). \\
\text{The arguments are sorted in increasing order with respect to cmp.}
}
\end{itemize}

\item[For listary generators]~
  \begin{itemize}
  \item
    \ift{commutative (cmp)}{
      C[...; x; y; ...] \rw C [...; y; x; ...] ~~\text{if {\sf cmp x y > 0}} \\
      \text{The arguments are sorted in increasing order with respect to cmp.}
    }
  \end{itemize}

\item[For n-ary generators] \na

\end{description}


\subsection{Distributive}
\label{sec:distributive}

This is the longest specification of all due to the number of
arguments possible to the distributive keyword.

The first step towards a specification of the semantics of
distributive concerns the definition of the notion of arity
compatibility.

\begin{definition}[Arity compatibility]
The arity of $E$ is {\em compatible} with the arity of $D$ if $E$ is
listary (and thus can simulate any other arity) or if arity $(E) =$
arity $(D)$.
\end{definition}

\begin{remark}
distributive (D, E) is only defined when the arity of $E$ is
compatible with the arity of $D$.
  
\end{remark}

Distributive is defined as follows in \mysec{sec:bnf}:
\begin{alltt}
  distributive [<invopp>] [<side>] "(" constr-name ["," <constr_name>]  ")"
\end{alltt}
where:
\begin{itemize}
\item \verb&<invopp>& governs the order of the arguments of the
  generator $E$ in the right-hand side of the rules for
  distributivity (either same order as in $D$ or reversed).
  The rules for distributive inverse (D, E) are obtained by taking the
  definition for distributive (D, E) and writing the arguments of $E$
  in reverse order.

  If we have:  
  \begin{lstlisting}
    type t = private
    | E of int
    | Add of t * t
    | Mult of t * t
      begin
      distributive (Add}
      end
    \end{lstlisting}
    then
    \begin{alltt}
      Mult (x, Plus (y, z) -> Plus (Mult (x, y), Mult (x, z))
    \end{alltt}
    and if we have:
    \begin{lstlisting}
    type t = private
    | E of int
    | Add of t * t
    | Mult of t * t
      begin
      distributive inverse (Add}
      end
    \end{lstlisting}
    then
    \begin{alltt}
      Mult (x, Plus (y, z) -> Plus (Mult (x, z), Mult (x, y))
    \end{alltt}
\end{itemize}

Let us now give the specifications for distributivity:

\begin{description}

\item[For zero-ary generators] \na
\item[For unary generators] 
\ift{distributive (D, E)}{
  $if $E$ is zeroary (then $D$ must be too) $C (D) \rw E \\
  $if $E$ is unary (then $D$ must be too) $C (D (x)) \rw E (C (x)) \\
  $if $E$ is binary (then $D$ must be too) $C (D (x, y)) \rw E (C(x), C(y))
  \\
  $if $E$ is binary then if $D$ is$ \\
  \begin{array}{l}
    $zeroary:$  C(D) \rw E [] \\
    $unary:$ C(D (x)) \rw E [x] \\
    $binary:$ C(D (x, y)) \rw E [C(x); C(y)] \\
    $listary:$ C(D [x_1; ...; x_n]) \rw E [C(x_1); ...; C(x_n)] \\
    $nary:$ C(D (x_1, ..., x_n)) \rw E [C(x_1); ...; C(x_n)] \\
  \end{array} \\
  $if $E$ is nary (then $D$ must be too and of the same arity as well) $
  C (D (x_1, ..., x_n)) \rw E (C(x_1), ..., C(x_n))
}
\item[For binary generators] 


\item[For listary generators] 

\item[For n-ary generators] \na

\end{description}


\subsection{Idempotent}
\label{sec:idempotent}
Examples of idempotent elements:
\begin{itemize}
\item The identity function is idempotent.
\item The boolean and is also idempotent.
\end{itemize}

\begin{description}

\item[For zero-ary generators] \na.
\item[For unary generators] 
\ift{idempotent}{ C (C (x)) \rw C (x)}
idempotent left and idempotent right are errors.
\item[For binary generators] 
\ift{idempotent left}
{ C (C (x, y), y) \rw C (x, y) \\
  C (x , x) \rw x \text{ if } x \text{ and } C(x, x) \text{ have the
    same type}. 
}
\ift{idempotent right}
{ C (x, C (x, y)) \rw C (x, y) \\
  C (x , x) \rw x \text{ if } x \text{ and } C(x, x) \text{ have the
    same type}. 
}
idempotent is treated as the conjunction of idempotent left and
idempotent right.

\item[For listary generators] 
\ift{idempotent}
{C[...; x; x; ...] \rw C [... ; x; ...] \\
 C[x; x]\rw x  \text{ if } x \text{ and } C[x; x] \text{ have the same type}
}

idempotent, idempotent left and idempotent right are identical.
\item[For n-ary generators] \na

\end{description}

\subsection{Inverse}
\label{sec:inverse}
\begin{description}

\item[For zero-ary generators] \na
\item[For unary generators] 
\ift{inverse left (I)}{I(C(x))\rw x}
\ift{inverse right (I)}{C(I(x))\rw x}
inverse (I) is the conjunction of inverse left (I) and inverse right
(I).

\ift{inverse left (I, A)}{I(C(x))\rw A}
\ift{inverse right (I, A)}{C(I(x))\rw A}
inverse (I, A) is the conjunction of inverse left (I, A) and inverse
right (I, A).

\item[For binary generators] 
 If \C has inverse side (I) then \C must also have neutral
side (E) (see \mysec{sec:neutral}) and inverse side (I) defaults to
inverse side (I, E).

\ift{inverse left (I, A)}
{ I \text{implicitly has involutive (see \mysec{sec:involutive})}. \\
  C \text{implicitly has distributive left (E, A) (see
    \mysec{sec:distributive})}. \\ 
  C (I(x), x) \rw A
}

\ift{inverse right(I, A)}
{ I \text{implicitly has involutive (see \mysec{sec:involutive})}. \\
  C \text{implicitly has distributive right (E, A) (see
    \mysec{sec:distributive})}. \\ 
  C (x, I (x)) \rw A
}

inverse (I, A) is the conjunction of inverse left (I, A) and inverse
right (I, A). 
\item[For listary generators] If \C has inverse side (I) then \C must also have neutral
side (E) (see \mysec{sec:neutral}) and inverse side (I) defaults to
inverse side (I, E).

\ift{inverse left (I, A)}
{ I \text{implicitly has involutive (see \mysec{sec:involutive})}s. \\
  C \text{implicitly has distributive left (E, A) (see
    \mysec{sec:distributive})}. \\ 

            C [...; I (x); x; ...] \rw C [...; A; ...] \text{if
              the list of arguments has more than 2 elements} \\
            C [I (x); x] \rw A
  }

\ift{inverse right(I, A)}
{ I \text{implicitly has involutive (see \mysec{sec:involutive})}. \\
  C \text{implicitly has distributive right (E, A) (see
    \mysec{sec:distributive}) }. \\ 
    C [...; x; I (x); ...] \rw C [...; A; ...] \text{if
              the list of arguments has more than 2 elements} \\
            C [x; I (x)] \rw A
}

inverse (I, A) is the conjunction of inverse left (I, A) and inverse
right (I, A). 

\item[For n-ary generators] \na

\end{description}




\subsection{Involutive}
\label{sec:involutive}
The boolean negation $\neg$  is involutive\footnote{in classical logic}.
The unary minus sign $-$ in arithmetic is also involutive.

\begin{description}

\item[For zero-ary generators] \na
\item[For unary generators] 
\ift{involutive}{C(C(x))\rw x}
\item[For binary generators] \na
\item[For listary generators] \na
\item[For n-ary generators] \na

\end{description}








\subsection{Neutral}
\label{sec:neutral}

Examples:
\begin{itemize}
\item $0$ is the neutral element for $+$ for reals and integers.
\item $1$ is neutral for $*$ for reals and integers
\item true is neutral for the boolean or $\vee$
\end{itemize}

\begin{description}
\item[For zero-ary generators] \na
\item[For unary generators] 
\ift{neutral (E)}{
\text{It means that }E \text{ is a fixpoin for } C. \\
C(E) \rw E \\
\text{In the unary setting, neutral left and neutral right are errors.}
}
\item[For binary generators] 
\ift{neutral left (E)}{
C(E, x)\rw x
}
\ift{neutral right (E)}{
C(x, E)\rw x
}
neutral (E) is the conjunction of neutral left (E) and neutral right (E).
\item[For listary generators] 
\ift{neutral left (E)}{
C[...; E; x; ...]\rw C [...; x; ...] \\
C [x] \rw x \\
C [] \rw []
}
\ift{neutral right (E)}{
C[...; x; E; ...]\rw C [...; x; ...] \\
C [x] \rw x \\
C [] \rw []
}
neutral (E) is the conjunction of neutral left (E) and neutral right (E).

\item[For n-ary generators] \na

\end{description}


\subsection{Nilpotent}
\label{sec:nilpotent}
\begin{description}

\item[For zero-ary generators] \na.
\item[For unary generators] 
\ift{nilpotent (A)}{ C (C (x)) \rw A}
nilpotent left and nilpotent right are errors.
\item[For binary generators] 
\ift{nilpotent left (A)}
{ C (C (x, y), y) \rw C (x, A) \\
  C (x , x) \rw A
}
\ift{nilpotent right}
{ C (x, C (x, y)) \rw C (A, y) \\
  C (x , x) \rw A
}
nilpotent (A) is treated as the conjunction of nilpotent left (A) and
nilpotent right (A).

\item[For listary generators] 
\ift{nilpotent (A)}
{C[...; x; x; ...] \rw C [... ; A; ...] \\
 C[x; x]\rw A
}

nilpotent, nilpotent left and nilpotent right are identical.
\item[For n-ary generators] \na

\end{description}


\subsection{Rules}
\label{sec:rules}

User can write arbitrary rewrite rules using the same syntax as for
\ocaml pattern matchings. Examples of such rules can be found in
\mysec{chap:examples}.

\subsection{Completion }
\label{sec:completion}

\moca has the ability to complete user-defined equational theories,
making them convergent (i.e. confluent and noetherian). 
It uses the well-known Knuth-Bendix completion
procedure \cite{DKnuPBen70}. 

This is made possible using two kinds of annotation on generators:
precedence and status.


\section{Priorities}
\label{sec:priorities}

The priorities of \moca keywords with respect to one another are quite
essential to know which code should be generated. Actually, is states
what should be reduced first in order to obtain the canonical
form. These priorities reflect \moca's reduction strategy


\todo{State current priorities list of Moca}
