\usepackage{amsmath}
\usepackage{latexsym}
\usepackage{multirow}
\usepackage{geometry}
\usepackage{xspace, alltt}
\usepackage{fullpage}
%\usepackage{hevea}
\usepackage{ifthen}
\usepackage[T1]{fontenc}
\usepackage[latin1]{inputenc}
\usepackage[a4paper,pdftex,pdfstartview=FitH]{hyperref}
\usepackage{amssymb}
\usepackage{color}
\usepackage{graphicx}
\usepackage[leftbars]{changebar}
\usepackage[english]{babel}
\usepackage{fancyhdr}
\usepackage{titlesec}
\usepackage{amsmath,amssymb,amsthm}
\theoremstyle{plain} 
\newtheorem{definition}{Definition}
\theoremstyle{remark}
\newtheorem{remark}{Note}

\def\bitem{\item[$\bullet$]}

\def\moca{{\sf Moca}\xspace}
\def\zenon{{\sf Zenon}\xspace}
\def\focal{{\sf Focal}\xspace}
\def\ocaml{{\sf OCaml}\xspace}
\def\coq{{\sf Coq}\xspace}
\def\tamed{{\sf TaMed}\xspace}
\def\why{{\sf Why}\xspace}
\def\zenonweb{{\sf http://focal.inria.fr/zenon}\xspace}
\def\mocaweb{{\sf http://moca.inria.fr/}\xspace}
\def\mocacvs{{\sf http://camlcvs.inria.fr/cgi-bin/cvsweb/bazar-ocaml/moca/}\xspace}
\def\vnumber{{\sf 0.5.0}}
\def\C{$C$\xspace}

\newcommand{\todo}[1]{\hspace*{-3cm}
  \begin{minipage}{1.0\linewidth}
    {\bf TODO: \\ #1}
  \end{minipage}
}

\newcommand{\note}[1]{{\bf NOTE:} #1}
\newcommand{\ift}[2]{If \C has #1 then
\[
  \begin{array}{ll}
   #2 
  \end{array}
\]
}

\newcommand{\chaptitle}[1]{
\part*{#1}
\setcounter{section}{0}
\addcontentsline{toc}{part}{\protect{#1}}
\def\chapname{#1}
}

\newcommand{\secpage}[1]{section \ref{#1}, p.\pageref{#1}}
\newcommand{\mysec}[1]{Sec.\ref{#1}\xspace}

\newcommand{\pagetitle}[2]{
\vspace*{3cm}
\thispagestyle{empty}
\begin{center}
  {\Huge\bf #1}
\end{center}

\vspace*{2cm}
\begin{center}
  {\Large #2}
\end{center}
\vspace*{1cm}
\begin{center}
  {\Large Richard Bonichon}
\end{center}
\vfill
\def\postes{#2}
}


\usepackage{listings}

% \lstdefinelanguage{Moca}[Objective]{Caml}
% {morekeywords={rule, associative, commutative, nilpotent, inverse,
%     opposite, neutral, idempotent, distributive, absorbing, absorbent,
%   left, right, involutive, status, precedence, completion},%
% }%

\lstloadlanguages{[Objective]Caml,Moca}

\definecolor{lstbg}{gray}{0.98} \definecolor{lstfg}{gray}{0.10}
\definecolor{lstrule}{gray}{0.6} \definecolor{lstnum}{gray}{0.4}
\definecolor{lsttxt}{rgb}{0.3,0.2,0.6}
\newcommand{\lstbrk}{\mbox{$\color{blue}\scriptstyle\cdots$}}

\def\lp@basic{
  \ifmmode\normalfont\mathtt\mdseries\scriptsize
  \else\normalfont\ttfamily\mdseries\scriptsize
  \fi
}

\def\lp@inline{
  \ifmmode\normalfont\mathtt\scriptstyle
  \else\normalfont\ttfamily\mdseries\small
  \fi}

\def\lp@keyword{}
\def\lp@special{\color{lstfg}}
\def\lp@comment{\normalfont\ttfamily\mdseries}
\def\lp@string{\color{lstfg}}
\def\lp@ident{}
\def\lp@number{\rmfamily\tiny\color{lstnum}}

\lstdefinestyle{moca-style}{%
  basicstyle={\ifmmode\normalfont\mathtt\scriptstyle
  \else\normalfont\ttfamily\mdseries\small
  \fi},%
  identifierstyle={},%
  commentstyle={\normalfont\ttfamily\mdseries},%
  keywordstyle={\ifmmode\mathsf\else\sffamily\fi},%
  keywordstyle=[2]\color{lstfg},%
  stringstyle={\color{lstfg}},%
  emphstyle={}\underbar,%
  showstringspaces=false,%
  mathescape=true,%
  numberstyle={\rmfamily\tiny\color{lstnum}},%
  xleftmargin=6ex,xrightmargin=2ex,%
  framexleftmargin=1ex,%
  frame=l,%
  framerule=1pt,%
  rulecolor=\color{lstrule},%
  backgroundcolor=\color{lstbg},%
  moredelim={*[s]{/*@}{*/}},%
  moredelim={*[l]{//@}},
  morecomment={[is]{//NOPP-BEGIN}{NOPP-END}},
  mathescape=true,
  escapechar=`
% breaklines is broken when using a inline and background
%  breaklines,prebreak={\lstbrk},postbreak={\lstbrk},breakindent=5ex %
}

\lstdefinestyle{moca}{
  language={[Objective]Caml},
  alsolanguage=Moca,
  style=moca-style
}

\lstset{
  language=Moca,
  style=moca,
  basicstyle={ \ifmmode\normalfont\mathtt\mdseries\scriptsize
  \else\normalfont\ttfamily\mdseries\scriptsize
  \fi},
  stepnumber=1,
  numbers=left
}

% \lstset{
%   language=Moca,
%   captionpos=b,
%   xleftmargin=10mm,
%   xrightmargin=10mm,
% %%  frame=lines,
%   columns=[c]{fixed},
% %%  numbers=right,
%   stepnumber=2,
%   numberfirstline=false
% }

\def\mlm{{\sf .mlm}\xspace}
\def\mlms{{\sf .mlms}\xspace}
\def\cons{{\sf cons}\xspace}
\def\insertf{{\sf insert}\xspace}
\def\delete{{\sf delete}\xspace}
\def\potent{{\sf potent}\xspace}
\def\return{{\sf return}\xspace}
\def\compare{{\sf compare}\xspace}
\def\mrule{{\sf rule}\xspace}
\def\testd{{\sf test}\xspace}
\def\na{{\it not applicable}.}
\def\rw{\longrightarrow}
\newcommand{\ret}[1]{{\sf return\_#1}\xspace}

\newenvironment{foo}[0]{~\begin{\itemize}}{\end{itemize}}