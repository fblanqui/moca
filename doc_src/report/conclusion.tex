\chapter{Conclusion}

This report concludes what has been for an exceptionally fun year
working on \moca. I hope to be able to contribute in the future as
well as I still can right now. 

This report does not list all the bug-fixes I did during this year but
only the major features I worked on, which were
documentation/specification, listary generators and finding better
examples for \moca. This is also because I fixed a lot of bugs I
introduced myself.

I would like to warmly thank both Fr�d�ric and Pierre for giving the
chance to work with them. I learned a lot from both of you. It was a
pleasure to work in our small but very lighthearted and merry project
mostly with Laura, Cody and my thesis co-advisor Damien.


\section{Upcoming features}
\label{sec:upcoming}

Work is still being done for the coming {\sf 0.6} version. This
release will include the updated listary code generation of
\mysec{sec:list_current} as well as the ususal bug fixes.

In the distant to not-so-distant future, we hope to implement some (or
even all!) of the  following features, in no particular order:
\begin{itemize}
\item Memoization for shared data types. This might also influence
  how listary generators are treated: it must be made as easy as
  possible to memoize the collection of small specialized functions of
  \mysec{sec:arch}.
\item AC-Completion (emulation)
\item Relational types for abstract data types
\item Modular automated proofs for generated code 
\item Better automated generation of {\sf .ml} and {\sf .mli} from
  {\sf .mlm} file with keywords for the functions one would like to be
  visible from the interface file.
\item Exhaustivity of pattern-matching with relational data types is
  not the same as in \ocaml as some patterns cannot be obtained by
  construction: check for false alarms of non-exhaustivity with
  respect to \mlm specification.
\item Relational data types for \focal.
\end{itemize}

\section{And now ?}
\label{sec:conc}

Here we are at the end of this tutorial. Hopefully you can now develop
your own programs using \moca and \ocaml.

We have not talked about some of the optional flags that can be
activated on the command-line. They are described in the man page. Try
them !


%%% Local Variables: 
%%% mode: latex
%%% TeX-master: "main"
%%% End: 
