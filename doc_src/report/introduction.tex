
\chapter{Introduction}
\label{sec:intro}

\section{What is \moca ?}
\moca\ is a general construction functions generator for \ocaml\ data types
with invariants.

\moca\ allows the high-level definition and automatic management of
complex invariants for data types. In addition, \moca\ provides the
automatic generation of maximally shared values, independently or in
conjunction with the declared invariants.

A relational data type is a concrete data type that declares
invariants or relations that are verified by its constructors. For
each relational data type definition, \moca\ compiles a set of
construction functions that implements the declared relations.

\moca\ supports two kinds of relations:
\begin{itemize}
\item algebraic relations (such as associativity or commutativity of a
binary constructor), 
\item general rewrite rules that map some pattern constructors
of \moca\ and variables to some arbitrary user's define expression.
\end{itemize}

Algebraic relations are primitive, so that \moca\ ensures the
correctness of their treatment. By contrast, the general rewrite rules
are under the programmer's responsibility, so that the desired
properties must be verified by a programmer's proof before compilation
(including for completeness, termination, and confluence of the
resulting term rewriting system).

Algebraic invariants are specified by using keywords denoting
equational theories like commutativity and associativity. \moca\
generates construction functions that allow each equivalence class to
be uniquely represented by their canonical value.

For theoretical details, have a look at \cite{moca07} which should be
included in the distribution.

\section{Building \moca}

First download the last stable version from \mocaweb. At the time of
the writing of this report, the latest release is version \vnumber.

For developers or those who want to keep track of incoming features,
 do a checkout of the current cvs archive as follows\footnote{The line {\sf export
    CVSROOT=:pserver:anoncvs@camlcvs.inria.fr:/caml} works with {\sf
    bash} or {\sf zsh}. Read your shell manual if you do not know how
  to set environment variables with your shell.}:

\begin{verbatim}
export CVSROOT=:pserver:anoncvs@camlcvs.inria.fr:/caml
cvs login (* hit the enter key at the CVS password prompt*)
cvs checkout bazar-ocaml/moca
\end{verbatim}

Whatever you have done in the previous step, read {\em carefully} the
README and INSTALL files even if it will most probably be enough to
type the usual: 
\begin{verbatim}
./configure
make
make install
\end{verbatim}

The BSD-style man page of \moca is intended to be a handy full documentation of
the software once you have mastered it enough. Do not hesitate to use
it as well!

\section{Features}
\label{sec:features}

As of version \vnumber\ the following features are available:

For the upcoming versions the development will focus on the following themes:
\begin{itemize}
\item Completion
\item Limited strategies
\item Sharing
\item Automatic test generation
\end{itemize}


\section{Command line switches}
\label{sec:clswitches}

As of \moca~\vnumber, the following command line options are available: 
\begin{verbatim}
Usage: mocac [options] <.mlm[s] file>
  -c              generate a module for the input file argument
  -help           print this option list and exit
  --help          print this option list and exit
  -i              generate a module interface for the input file argument
  -I <dir>        add a directory to the search path
  -kb             set completion on (experimental option under development)
  -kb-limit <n>   set the upper bound to completion steps
  -ntests <n>     set the number of generated tests per equation
      (use with -test)
  -o <file>       output the module with name the file name argument
  -oml <file>     output the module implementation in the file argument
  -omli <file>    output the module interface in the file argument
  -otest <file>   output tests in the file argument
  -seed <n>       set the seed for random tests
      (use with -test)
  --sharing       generate maximally sharing data structures
  -test           generate a file for testing the code generated by Moca
  -urorder        use the relation order provided in the source file
  -valdepth <n>   set the maximal constructor depth of the values 
      generated for testing (to be used with -test)
  --verbose       be verbose during generation
  -v              print version string and exit
  --version       print full version string and exit
  -d              debugging mode

\end{verbatim}
