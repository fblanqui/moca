%%%%%%%%%%%%%%%%%%%%%%%%%%%%%%%%%%%%%%%%%%%%%%%%%%%%%%%%%%%%%%%%%%%%%%%%%%%%%%
% title
%%%%%%%%%%%%%%%%%%%%%%%%%%%%%%%%%%%%%%%%%%%%%%%%%%%%%%%%%%%%%%%%%%%%%%%%%%%%%%

\title{On the implementation of construction functions for non-free concrete data types}

\author{Fr\'ed\'eric Blanqui\inst{1}
\and Th\'er\`ese Hardin\inst{2}
\and Pierre Weis\inst{3}}

\institute{INRIA \& LORIA, BP 239,
54506 Villers-l\`es-Nancy Cedex, France
\and UPMC,
LIP6, 104, Av. du Pr. Kennedy, 75016 Paris, France
\and INRIA, Domaine de Voluceau, BP 105,
78153 Le Chesnay Cedex, France}

\maketitle

\begin{abstract}
Many algorithms use concrete data types with some additional invariants. The
set of values satisfying the invariants is often a set of representatives for
the equivalence classes of some equational theory. For instance, a sorted list
is a particular representative wrt commutativity. Theories like associativity,
neutral element, idempotence, etc. are also very common. Now, when one wants to
combine various invariants, it may be difficult to find the suitable
representatives and to efficiently implement the invariants. The preservation
of invariants throughout the whole program is even more difficult and error
prone. Classically, the programmer solves this problem using a combination of
two techniques: the definition of appropriate construction functions for the
representatives and the consistent usage of these functions ensured via
compiler verifications. The common way of ensuring consistency is to use an
abstract data type for the representatives; unfortunately, pattern matching on
representatives is lost. A more appealing alternative is to define a concrete
data type with private constructors so that both compiler verification and
pattern matching on representatives are granted. In this paper, we detail the
notion of private data type and study the existence of construction
functions. We also describe a prototype, called \moca, that addresses the
entire problem of defining concrete data types with invariants: it generates
efficient construction functions for the combination of common invariants and
builds representatives that belong to a concrete data type with private
constructors.
\end{abstract}
