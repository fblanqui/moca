%%%%%%%%%%%%%%%%%%%%%%%%%%%%%%%%%%%%%%%%%%%%%%%%%%%%%%%%%%%%%%%%%%%%%%%%%%%%%%
% Relational data types
%%%%%%%%%%%%%%%%%%%%%%%%%%%%%%%%%%%%%%%%%%%%%%%%%%%%%%%%%%%%%%%%%%%%%%%%%%%%%%

\section{Relational data types}
\label{sec-rdt}

We mentioned in the introduction that, often, the invariants upon
concrete data types are such that the set of values satisfying them is
indeed a set of representatives for the equivalence classes of some
equational theory. We therefore propose to specify invariants by a set
of unoriented equations and study to which extent such a specification
can be realized with an abstract or private data type. In case of a
private data type however, it is important to be able to describe the
set of possible values.

\begin{dfn}[Relational data type]
A {\em relational data type (RDT) definition} is a pair $(\G,\cE)$
where $\G=(\pi,\cC,\S)$ is a CDT definition and $\cE$ is a finite set
of equations on $\cT_\pi(\cA_\G,\cX)$. Let $=_\cE$ be the smallest
congruence relation containing $\cE$. Such an RDT is {\em
implementable} by a PDT $(\G,\cF)$ if the family of construction
functions $\cF=(f_C)_{C\in\cC}$ is {\em valid wrt $\cE$}:

\begin{lst}{--}
\item [\bf(Correctness)]
For all $C:\s_1..\s_n\pi$ and $v_i\in Val(\s_i)$,
$f_C(v_1..v_n)=_\cE C(v_1..v_n)$.
\item [\bf(Completeness)]
For all $C:\s_1..\s_n\s$, $v_i\in Val(\s_i)$,
$D:\tau_1..\tau_p\s\in\S$ and $w_i\in Val(\tau_i)$, $f_C(v_1..v_n)=
f_D(w_1..w_p)$ whenever $C(v_1..v_n)=_\cE D(w_1..w_p)$.
\end{lst}
\end{dfn}

%%%%%%%%%%%%%%%%%%%%%%%%%%%%%%%%%%%%%%%%%%%%%%%%%%%%%%%%%%%%%%%%%%%%%%%%%%%%%%
% Valid normalization function
%%%%%%%%%%%%%%%%%%%%%%%%%%%%%%%%%%%%%%%%%%%%%%%%%%%%%%%%%%%%%%%%%%%%%%%%%%%%%%

We are going to see that the existence of a valid family of
construction functions is equivalent to the existence of a valid
normalization function:

\begin{dfn}[Valid normalization function]
A map $f:\cT_\pi(\cA_\G)\a\cT_\pi(\cA_\G)$ is a {\em valid
normalization function} for an RDT $(\G,\cE)$ with $\G=(\pi,\cC,\S)$
if:

\begin{lst}{--}
\item [\bf(Correctness)]
For all $t\in\cT_\pi(\cA_\G)$, $f(t)=_\cE t$.
\item [\bf(Completeness)]
For all $t,u\in\cT_\pi(\cA_\G)$, $f(t)= f(u)$ whenever $t=_\cE u$.
\end{lst}
\end{dfn}

Note that a valid normalization function is idempotent ($f\circ f=f$)
and provides a decision procedure for $=_\cE$ (the boolean function
$\la xy.f(x)= f(y)$).

%%%%%%%%%%%%%%%%%%%%%%%%%%%%%%%%%%%%%%%%%%%%%%%%%%%%%%%%%%%%%%%%%%%%%%%%%%%%%%
% family -> normalization function
%%%%%%%%%%%%%%%%%%%%%%%%%%%%%%%%%%%%%%%%%%%%%%%%%%%%%%%%%%%%%%%%%%%%%%%%%%%%%%

\begin{thm}
\label{thm-f}
The normalization function associated to a valid family is a valid
normalization function.
\end{thm}

\begin{prf}
\begin{lst}{--}
\item Correctness. We proceed by induction on the size of $t\in\cT_\pi$.
We have $C:\s_1..\s_n\pi\in\S$ and $t_i$ such that $t=C(t_1..t_n)$. By
definition, $f(t)=f_C(f(t_1)..$ $f(t_n))$. By induction hypothesis,
$f(t_i)=_\cE t_i$. Since the family is valid and $f(t_1)..f(t_n)$ are
values, $f_C(f(t_1)..f(t_n))=_\cE C(f(t_1)..f(t_n))$. Thus, $f(t)=_\cE
t$.

\item Completeness.
Let $t,u\in\cT_\pi$ such that $t=_\cE u$. We have $t=C(t_1..t_n)$ and
$u=D(u_1..u_p)$. By definition, $f(t)=f_C(f(t_1)..f(t_n))$ and
$f(u)=f_D(f(u_1)..f(u_p))$. By correctness, $f(t_i)=_\cE t_i$ and
$f(u_j)=_\cE u_j$. Hence, $C(f(t_1)..f(t_n))=_\cE
D(f(u_1)..f(u_p))$. Since the family is valid and $f(t_1)..f(t_n)$ are
values, $f_C(f(t_1)$ $..f(t_n))=f_D(f(t_1)..f(t_n))$. Thus,
$f(t)=f(u)$.\cqfd\\
\end{lst}
\end{prf}

%%%%%%%%%%%%%%%%%%%%%%%%%%%%%%%%%%%%%%%%%%%%%%%%%%%%%%%%%%%%%%%%%%%%%%%%%%%%%%
% normalization function -> family
%%%%%%%%%%%%%%%%%%%%%%%%%%%%%%%%%%%%%%%%%%%%%%%%%%%%%%%%%%%%%%%%%%%%%%%%%%%%%%

Conversely, given $f:\cT_\pi(\cA_\G)\a\cT_\pi(\cA_\G)$, one can
easily define a family of construction functions that is valid
whenever $f$ is a valid normalization function.

\begin{dfn}[Associated family of constr. functions]
Given a CDT $\G=(\pi,\cC,\S)$ and a function
$f:\cT_\pi(\cA_\G)\a\cT_\pi(\cA_\G)$, the {\em family of construction
functions associated to $f$} is the family $(f_C)_{C\in\cC}$ such
that, for all $C:\s_1..\s_n\pi\in\S$ and $t_i\in\cT_{\s_1}(\cA_\G)$,
$f_C(t_1, \ldots,t_n)=f(C(t_1, \ldots, t_n))$.
\end{dfn}

\begin{thm}
\label{thm-g}
The family of construction functions associated to a valid
normalization function is valid.
\end{thm}

\comment{
\begin{prf}
\begin{lst}{--}
\item Correctness.
Let $C:\s_1..\s_n\g\in\S$ and $v_i\in Val_\cF(\s_i)$. Since $f$ is
valid, we have $f_C(v_1, \ldots, v_n)=f(C(v_1, \ldots, v_n))=_\cE
C(v_1, \ldots, v_n)$.

\item Completeness.
Let $C:\s_1..\s_n\pi\in\S$, $v_i\in Val_\cF(\s_i)$,
$D:\tau_1..\tau_p\pi\in\S$ and $w_i\in Val_\cF(\s_i)$ such that
$C(v_1, \ldots, v_n)=_\cE D(w_1, \ldots, w_p)$. Since $f$ is valid,
$f_C(v_1, \ldots, v_n)=f(C(v_1, \ldots, v_n))= f(D(w_1, \ldots,
w_p))=f_D(w_1, \ldots, w_p)$.\cqfd
\end{lst}
\end{prf}
}

\begin{expl}
We can choose {\tt cexp} as the underlying CDT and $\cE = \{$ {\tt
Plus x Zero = x}$\}$ to define a RDT implementable by the PDT {\tt
exp}, with the valid family of construction functions {\tt zero}, {\tt
one}, {\tt opp}, {\tt plus}.
\end{expl}
