%%%%%%%%%%%%%%%%%%%%%%%%%%%%%%%%%%%%%%%%%%%%%%%%%%%%%%%%%%%%%%%%%%%%%%%%%%%%%%
% Relational data types
%%%%%%%%%%%%%%%%%%%%%%%%%%%%%%%%%%%%%%%%%%%%%%%%%%%%%%%%%%%%%%%%%%%%%%%%%%%%%%

\section{Concrete data types with private constructors}
\label{sec-pdt}

%%%%%%%%%%%%%%%%%%%%%%%%%%%%%%%%%%%%%%%%%%%%%%%%%%%%%%%%%%%%%%%%%%%%%%%%%%%%%%
% First-order term algebra
%%%%%%%%%%%%%%%%%%%%%%%%%%%%%%%%%%%%%%%%%%%%%%%%%%%%%%%%%%%%%%%%%%%%%%%%%%%%%%

We first recall the definition of a first-order term algebra. It will
be useful for defining the values of concrete and private data types.

\begin{dfn}[First-order term algebra]
A {\em sorted term algebra definition} is a triplet $\cA=(\cS,\cC,\S)$
where $\cS$ is a non-empty set of {\em sorts}, $\cC$ is a non-empty
set of {\em constructor symbols} and $\S:\cC\a\cS^+$ is a {\em
signature} mapping a non-empty sequence of sorts to every constructor
symbol. We write $C:\s_1 \ldots\s_n\s_{n+1}\in\S$ to denote the fact
that $\S(C)=\s_1 \ldots\s_n\s_{n+1}$. Let $\cX=(\cX_\s)_{\s\in\cS}$ be
a family of pairwise disjoint sets of {\em variables}. The sets
$\cT_\s(\cA,\cX)$ of {\em terms of sort $\s$} are inductively defined
as follows:

\begin{lst}{--}
\item If $x\in\cX_\s$, then $x\in\cT_\s(\cA,\cX)$.
\item If $C:\s_1\ldots\s_{n+1}\in\S$ and
$t_i\in\cT_{\s_i}(\cA,\cX)$, then
${C(t_1,\ldots,t_n)}\in\cT_{\s_{n+1}}(\cA,\cX)$.
\end{lst}

\noindent
Let $\cT_\s(\cA)$ be the set of terms of sort $\s$ containing no
variable.% We assume that $\cT_\s(\cA)\neq\vide$.
\end{dfn}

In the following, we assume given a set $\cS_0$ of primitive types
like {\tt int}, {\tt string}, \ldots and a set $\cC_0$ of primitive
constants {\tt 0}, {\tt 1}, {\tt "foo"}, \ldots  Let $\S_0$ be the
corresponding signature ($\S_0({\tt 0})={\tt int}$, \ldots).

%%%%%%%%%%%%%%%%%%%%%%%%%%%%%%%%%%%%%%%%%%%%%%%%%%%%%%%%%%%%%%%%%%%%%%%%%%%%%%
% concrete data types
%%%%%%%%%%%%%%%%%%%%%%%%%%%%%%%%%%%%%%%%%%%%%%%%%%%%%%%%%%%%%%%%%%%%%%%%%%%%%%

In this paper, we call {\em concrete data type} (CDT) an inductive
type {\em\`a la} ML defined by a set of {\em constructors}. More
formally:

\begin{dfn}[Concrete data type]
A {\em concrete data type definition} is a triplet $\G=(\g,\cC,\S)$
where $\g$ is a sort, $\cC$ is a non-empty set of {\em constructor
symbols} and $\S:\cC\a(\cS_0\cup\{\g\})^+$ is a {\em signature} such
that, for all $C\in\cC$, $\S(C)=\s_1..\s_n\g$. The set $Val(\g)$ of
{\em values of type $\g$} is the set of terms $\cT_\g(\cA_\G)$ where
$\cA_\G=(\cS_0\cup\{\g\},\cC_0\cup\cC,\S_0\cup\S)$.
\end{dfn}

This definition of CDTs corresponds to a small but very useful subset
of all the possible types definable in ML-like programming
languages. For the purpose of this paper, it is not necessary to use a
more complex definition.

\begin{expl}
The following type\footnote{Examples are written with
\ocaml\ \cite{ocaml3.09}, they can be readily translated in any
programming language offering pattern-matching with textual priority,
as Haskell, SML, etc.} {\tt cexp} is a CDT definition with two
constant constructors of sort {\tt cexp} and a binary operator of sort
{\tt cexp} {\tt cexp} {\tt cexp}.

{\small\begin{verbatim}
type cexp =  Zero | One | Opp of cexp | Plus of cexp * cexp
\end{verbatim}}
\end{expl}

%%%%%%%%%%%%%%%%%%%%%%%%%%%%%%%%%%%%%%%%%%%%%%%%%%%%%%%%%%%%%%%%%%%%%%%%%%%%%%
% private data types
%%%%%%%%%%%%%%%%%%%%%%%%%%%%%%%%%%%%%%%%%%%%%%%%%%%%%%%%%%%%%%%%%%%%%%%%%%%%%%

Now, a private data type definition is like a CDT definition together
with construction functions as in abstract data types. Constructors
can be used as patterns as in concrete data types but they {\em
cannot} be used for value creation (except in the definition of
construction functions). For building values, one must use
construction functions as in abstract data types. Formally:

\begin{dfn}[Private data type]
A {\em private data type definition} is a pair $\Pi=(\G,\cF)$ where
$\G=(\pi,\cC,\S)$ is a CDT definition and $\cF$ is a family of {\em
construction functions} $(f_C)_{C\in\cC}$ such that, for all
$C:\s_1..\s_n\pi\in\S$,
$f_C:\cT_{\s_1}(\cA_\G)\times\ldots\times\cT_{\s_n}(\cA_\G)\a\cT_\pi(\cA_\G)$.
Let $Val(\pi)$ be the set of the {\em values of type $\pi$}, that is,
the set of terms that one can build by using the construction
functions only. The function $f:\cT_\pi(\cA_\G)\a\cT_\pi(\cA_\G)$ such
that, for all $C:\s_1..\s_n\pi\in\S$ and $t_i\in\cT_{\s_i}(\cA_\G)$,
$f(C(t_1..t_n))=f_C(f(t_1)..f(t_n))$, is called the {\em normalization
function associated to $\cF$}.
\end{dfn}

%%%%%%%%%%%%%%%%%%%%%%%%%%%%%%%%%%%%%%%%%%%%%%%%%%%%%%%%%%%%%%%%%%%%%%%%%%%%%%
% values of private data types
%%%%%%%%%%%%%%%%%%%%%%%%%%%%%%%%%%%%%%%%%%%%%%%%%%%%%%%%%%%%%%%%%%%%%%%%%%%%%%

This is quite immediate to see that:

\begin{lemma}
$Val(\pi)$ is the image of $f$.
\end{lemma}

%%%%%%%%%%%%%%%%%%%%%%%%%%%%%%%%%%%%%%%%%%%%%%%%%%%%%%%%%%%%%%%%%%%%%%%%%%%%%%
% implementation
%%%%%%%%%%%%%%%%%%%%%%%%%%%%%%%%%%%%%%%%%%%%%%%%%%%%%%%%%%%%%%%%%%%%%%%%%%%%%%

PDTs have been implemented in \ocaml\ by the third author
\cite{weis03}.  Extending a programming language with PDTs is not very
difficult: one only needs to modify the compiler to parse the PDT
definitions and check that the conditions on the use of constructors
are fulfilled.

Note that construction functions have no constraint in general: the
full power of the underlying programming language is available to
define them.

%%%%%%%%%%%%%%%%%%%%%%%%%%%%%%%%%%%%%%%%%%%%%%%%%%%%%%%%%%%%%%%%%%%%%%%%%%%%%%
% pattern matching
%%%%%%%%%%%%%%%%%%%%%%%%%%%%%%%%%%%%%%%%%%%%%%%%%%%%%%%%%%%%%%%%%%%%%%%%%%%%%%

It should also be noted that, because the set of values of type $\pi$
is a subset of the set of values of the underlying CDT $\g$, a
function on $\pi$ defined by pattern matching may be a total function
even though it is not defined on all the possible cases of
$\g$. Defining a function with patterns that match no value of type
$\pi$ does not harm since the corresponding code will never be run. It
however reveals that the developer is not aware of the distinction
between the values of the PDT and those of the underlying CDT, and
thus can be considered as a programming error. To avoid this kind of
errors, it is important that a PDT comes with a clear identification
of its set of possible values. To go one step further, one could
provide a tool for checking the completeness and usefulness of
patterns that takes into account the invariants, when it is
possible. We leave this for future work.

%%%%%%%%%%%%%%%%%%%%%%%%%%%%%%%%%%%%%%%%%%%%%%%%%%%%%%%%%%%%%%%%%%%%%%%%%%%%%%
% subset example
%%%%%%%%%%%%%%%%%%%%%%%%%%%%%%%%%%%%%%%%%%%%%%%%%%%%%%%%%%%%%%%%%%%%%%%%%%%%%%


\begin{expl}
Let us now start our running example with the type {\tt exp}
describing operations on arithmetic expressions.

{\small\begin{verbatim}
type exp = private Zero | One | Opp of exp | Plus of exp * exp
\end{verbatim}}

This type {\tt exp} is indeed a PDT built upon the CDT {\tt
cexp}. Prompted by the keyword {\tt private}, the \ocaml\ compiler
forbids the use of {\tt exp} constructors (outside the module {\tt
my\_exp.ml} containing the definition of {\tt exp}) except in
patterns.  If {\tt Zero} is supposed to be neutral by the writer of
{\tt my\_exp.ml}, then he/she will provide construction functions as
follows:

{\small\begin{verbatim}
let rec zero = Zero and one = One and opp x = Opp x
and plus = function
| (Zero,y) ->  y
| (y,Zero) -> y
| (x,y) -> Plus(x,y)
\end{verbatim}}
\end{expl}


\comment{Let us now give a very simple example showing how PDTs can be used for
representing subsets:\footnote{For our examples, we use the \ocaml\
\cite{ocaml3.09} syntax but they could easily be translated to other
programming languages.}

{\small\begin{verbatim}
(* interface file nat.mli *)
type nat = private Nat of int
val nat_of_int : int -> nat

(* implementation file nat.ml *)
type nat = Nat of int
let nat_of_int x = if x<0 then invalid_arg "negative" else Nat x

(* interface file array.mli *)
open Nat
type array
val array_make : nat -> string -> array

(* implementation file array.ml *)
open Nat
type array = ...
let int_of_nat (Nat x) = x
let array_make n s = ... (* here, we have [int_of_nat n >= 0] *)
\end{verbatim}}

A value $n$ of type {\tt nat} can be obtained only by using {\tt
int\_of\_nat} which ensures that $n$ is a non-negative integer. Note
that the test $n\ge 0$ is done only once, at value creation.
We think that it should be possible to get rid of the constructor {\tt
Nat} and allow declarations like {\tt type nat = private int}.
}