%%%%%%%%%%%%%%%%%%%%%%%%%%%%%%%%%%%%%%%%%%%%%%%%%%%%%%%%%%%%%%%%%%%%%%%%%%%%%%
% moca
%%%%%%%%%%%%%%%%%%%%%%%%%%%%%%%%%%%%%%%%%%%%%%%%%%%%%%%%%%%%%%%%%%%%%%%%%%%%%%

\section{The \moca\ system}
\label{sec-moca}

We now describe the \moca\ prototype, a program generator that
implements an extension of \ocaml\ with RDTs. \moca\ parses a special
``.mlm'' file containing the RDT definition and produces a regular
\ocaml\ module (interface and implementation) which provides the
construction functions for the RDT. \moca\ provides a set of keywords
for specifying the equations described in Figure \ref{fig-eq}.

For instance, the RDT {\tt exp} can be defined in \moca\ as follows:

{\small\begin{verbatim}
type exp = private Zero | One | Opp of exp | Plus of exp * exp
  begin associative commutative neutral(Zero) opposite(Opp) end
\end{verbatim}}

\moca\ also features user's arbitrary rules with the construction:
{\small {\tt rule} $pattern$ {\tt ->} $pattern$}. These rules add
extra clauses in the definitions of construction functions generated
by \moca: the LHS $pattern$ is copied verbatim as the pattern of a
clause which returns the RHS $pattern$ considered as an expression
where constructors are replaced by calls to the corresponding
construction functions. Of course, in the presence of such arbitrary
rules, we cannot guarantee the termination or completeness of the
generated code. This construction is thus provided for expert users
that can prove termination and completeness of the corresponding set
of rules. That way, the programmer can describe complex RDTs, even
those which cannot be described with the set of predefined equational
invariants.

\moca\ also accepts polymorphic RDTs and RDTs mutually defined with
record types (but equations between record fields are not yet
available).

The equations of Figure \ref{fig-eq} also support n-ary constructor,
implemented as unary constructors of type {\tt t list -> t}. In this
case, {\tt Plus} gets a single argument of type {\tt exp list}. Normal
forms are modified accordingly and use lists instead of combs. For
instance, associative normal forms get flat lists of arguments: in a
{\tt Plus}$(l)$ expression, no element of $l$ is a {\tt Plus}$(l')$
expression. The corresponding data structure is widely used in
rewriting.

Finally, \moca\ offers an important additional feature: it can
generate construction functions that provide maximally shared
representatives. To fire maximal sharing, just add the {\tt --sharing}
option when compiling the ``.mlm'' file. In this case, the generated
type is slightly modified, since every functional constructor gets an
extra argument to keep the hash code of the term. Maximally shared
representatives have a lot of good properties: not only data size is
minimal and user's memoized functions can be light speed, but
comparison between representatives is turned from a complex recursive
term comparison to a pointer comparison -- a single machine
instruction. \moca\ heavily uses this property for the generation of
construction functions: when dealing with non-linear equations, the
maximal sharing property allows \moca\ to replace term equality by
pointer equality.
