%%%%%%%%%%%%%%%%%%%%%%%%%%%%%%%%%%%%%%%%%%%%%%%%%%%%%%%%%%%%%%%%%%%%%%%%%%%%%%
% TOWARDS EFFICIENT CONSTRUCTION FUNCTIONS
%%%%%%%%%%%%%%%%%%%%%%%%%%%%%%%%%%%%%%%%%%%%%%%%%%%%%%%%%%%%%%%%%%%%%%%%%%%%%%

\section{Towards efficient construction functions}
\label{sec-genr}

%%%%%%%%%%%%%%%%%%%%%%%%%%%%%%%%%%%%%%%%%%%%%%%%%%%%%%%%%%%%%%%%%%%%%%%%%%%%%%
% without commutative symbols
%%%%%%%%%%%%%%%%%%%%%%%%%%%%%%%%%%%%%%%%%%%%%%%%%%%%%%%%%%%%%%%%%%%%%%%%%%%%%%

When there is no commutative symbol, construction functions can be
easily implemented by simulating innermost rewriting as follows:

\begin{dfn}[Linearization]
Let $\vpos(t)$ be the set of positions $p\in\pos(t)$ such that $t|_p$
is a variable $x\in\cX$. Let $\r:\vpos(t)\a\cX$ be an injective
mapping and $lin(t)$ be the term obtained by replacing in $t$ every
subterm at position $p\in\vpos(t)$ by $\r(p)$. Let now $Eq(t)$ be the
conjunction of {\tt true} and of the equations $\r(p)=\r(q)$ such that
$t|_p=t|_q$ and $p,q\in\vpos(t)$.
\end{dfn}

\begin{dfn}
\label{dfn-r}
Given a set $\cR$ of rewrite rules, let $\cF(\cR)$ be the family of
construction functions $(f_C)_{C\in\cC}$ defined as follows:

\begin{lst}{\bu}
\item For every rule $l\a r\in\cR$ with $l=C(l_1,\ldots,l_n)$,
add to the definition of $f_C$ the clause {\tt
$lin(l_1),\ldots,lin(l_n)$ when $Eq(l)$ -> $\h{lin(r)}$}, where
$\h{t}$ is the term obtained by replacing in $t$ every occurrence of a
constructor $C$ by a call to its construction function $f_C$.
\item Terminate the definition of $f_C$ by the {\em default clause}
{\tt x -> C(x)}.
\end{lst}
\end{dfn}

\begin{thm}
\label{thm-1}
Assume that $\cE_{AC}=\vide$ and $\cE$ has a complete presentation
$\cR$. Then, $\cF(\cR)$ is valid wrt $\cE$ (whatever the order of the
non-default clauses is).
\end{thm}

%%%%%%%%%%%%%%%%%%%%%%%%%%%%%%%%%%%%%%%%%%%%%%%%%%%%%%%%%%%%%%%%%%%%%%%%%%%%%%
% with commutative symbols
%%%%%%%%%%%%%%%%%%%%%%%%%%%%%%%%%%%%%%%%%%%%%%%%%%%%%%%%%%%%%%%%%%%%%%%%%%%%%%

We now consider the case of commutative symbols. We are going to
describe a modular way of defining the construction functions by
pursuing our running  example, with the type {\tt exp}. 
Assume that {\tt Plus} is declared to be associative
and commutative only. The construction functions can then be defined
as follows:

{\small\begin{verbatim}
let zero = Zero and one = One and opp x = Opp x

and plus = function
| Plus(x,y), z -> plus (x, plus (y,z))
| x, y -> insert_plus x y

and insert_plus x = function
| Plus(y,_) as u when x <= y -> Plus(x,u)
| Plus(y,t) -> Plus (y, insert_plus x t)
| u when x > u -> Plus(u,x)
| u -> Plus(x,u)
\end{verbatim}}

One can easily see that {\tt plus} does the same job as the function
$\sort\circ\comb$ used in Theorem \ref{thm-ac} but in a slightly more
efficient way since $\cA$-normalization and sorting are interleaved.

%%%%%%%%%%%%%%%%%%%%%%%%%%%%%%%%%%%%%%%%%%%%%%%%%%%%%%%%%%%%%%%%%%%%%%%%%%%%%%
% with neutral
%%%%%%%%%%%%%%%%%%%%%%%%%%%%%%%%%%%%%%%%%%%%%%%%%%%%%%%%%%%%%%%%%%%%%%%%%%%%%%

Assume moreover that {\tt Zero} is neutral. The AC-completion of
$\{$ {\tt Plus}$(${\tt Zero}$,x)$ $=x\}$ gives $\{$ {\tt Plus}$(${\tt
Zero}$,x)\a x\}$. Hence, if $x$ and $y$ are terms in normal form, then
{\tt Plus}$(x,y)$ can be rewritten modulo AC only if $x=$ {\tt Zero}
or $y= $ {\tt Zero}. Thus, the function {\tt plus} needs to be
extended with two new clauses only:

{\small\begin{verbatim}
and plus = function
| Zero, y -> y
| x, Zero -> x
| Plus(x,y), z -> plus (x, plus (y,z))
| x, y -> insert_plus x y
\end{verbatim}}

%%%%%%%%%%%%%%%%%%%%%%%%%%%%%%%%%%%%%%%%%%%%%%%%%%%%%%%%%%%%%%%%%%%%%%%%%%%%%%
% inverse
%%%%%%%%%%%%%%%%%%%%%%%%%%%%%%%%%%%%%%%%%%%%%%%%%%%%%%%%%%%%%%%%%%%%%%%%%%%%%%

Assume now that {\tt Plus} is declared to have {\tt Opp} as
inverse. Then, the completion modulo AC of
$\{$ {\tt Plus}$(${\tt Zero}$,x)=x,$
{\tt Plus}$(${\tt Opp}$(x),x)=$ {\tt Zero}$\}$ 
gives the following well known rules for
abelian groups \cite{hullot80thesis}: 
$\{$ {\tt Plus}$(${\tt Zero}$,x)\a x$,\quad
{\tt Plus}$(${\tt Opp}$(x),x)\a $ {\tt Zero},\quad
{\tt Plus}$(${\tt Plus}$(${\tt Opp}$(x),x),y)\a y$,\quad
{\tt Opp}$(${\tt Zero}$)\a${\tt Zero},\quad
{\tt Opp}$(${\tt Opp}$(x))\a x$,\quad
{\tt Opp}$(${\tt Plus}$(x,y))\a$
{\tt Plus}$(${\tt Opp}$(y),${\tt Opp}$(x))\,\}$.

The rules for {\tt Opp} are easily translated as follows:

{\small\begin{verbatim}
and opp = function
| Zero -> Zero
| Opp(x) -> x
| Plus(x,y) -> plus (opp y, opp x)
| _ -> Opp(x)
\end{verbatim}}

The third rule of abelian groups is called an {\em extension} of the
second one since it is obtained by first adding the context
$Plus([],y)$ on both sides of this second rule,then normalizing the
right hand side. Take now two terms $x$ and $y$ in normal form and
assume that $(x,y)$ matches none of the three clauses previously
defining {\tt plus}, that is, $x$ and $y$ are distinct from {\tt
Zero}, and $x$ is not of the form {\tt Plus}$(x_1,x_2)$. To get the
normal form of {\tt Plus}$(x,y)$, we need to check that $x$ and the
normal form of its opposite {\tt Opp}$(x)$ do not occur in $y$. The
last clause defining {\tt plus} needs therefore to be modified as
follows:

{\small\begin{verbatim}
and plus = function
| Zero, y -> y
| x, Zero -> x
| Plus(x,y), z -> plus (x, plus (y,z))
| x, y -> insert_opp_plus (opp x) y

and insert_opp_plus x y =
  try delete_plus x y
  with Not_found -> insert_plus (opp x) y

and delete_plus x = function
| Plus(y,_) when x < y -> raise Not_found
| Plus(y,t) when x = y -> t
| Plus(y,t) -> Plus (y, delete_plus x t)
| y when y = x -> Zero
| _ -> raise Not_found
\end{verbatim}}

Forgetting about {\tt Zero} and {\tt Opp}, suppose now that {\tt Plus}
is declared associative, commutative and idempotent. The function {\tt
plus} is kept but the {\tt insert} function is modified as follows:

{\small\begin{verbatim}
and insert_plus x = function
| Plus(y,_) as u when x = y -> u
| Plus(y,_) as u when x < y -> Plus(x,u)
| Plus(y,t) -> Plus (y,insert_plus x t)
| u when x > u -> Plus(u,x)
| u when x = u -> u
| u -> Plus(x,u)
\end{verbatim}}

Nilpotence can be dealt with in a similar way.

%%%%%%%%%%%%%%%%%%%%%%%%%%%%%%%%%%%%%%%%%%%%%%%%%%%%%%%%%%%%%%%%%%%%%%%%%%%%%%
% theorem
%%%%%%%%%%%%%%%%%%%%%%%%%%%%%%%%%%%%%%%%%%%%%%%%%%%%%%%%%%%%%%%%%%%%%%%%%%%%%%

In conclusion, for various combinations of the equations of Figure
\ref{fig-eq}, we can define in a nice modular way construction
functions that are more efficient than the ones based on rewriting
modulo AC. We summarize this as follows:

\begin{dfn}
A set of equations $\cE$ is a theory of type:
\begin{enumi}{}
\item if $\cE_{AC}=\vide$ and $\cE$ has a complete presentation,
\item if $\cE$ is the union of $\{Assoc(C),Com(C)\}$
with either $\{Neu(C,E),Inv(C,I,E)\}$,
$\{Idem(C)\}$, $\{Neu(C,E),Idem(C)\}$
$\{Nil(C,A)\}$ or $\{Neu(C,E),Nil(C,A)\}$.
\end{enumi}

\noindent
Two theories are disjoint if they share no symbol.
\end{dfn}

%%%%%%%%%%%%%%%%%%%%%%%%%%%%%%%%%%%%%%%%%%%%%%%%%%%%%%%%%%%%%%%%%%%%%%%%%%%%%%
% construction functions
%%%%%%%%%%%%%%%%%%%%%%%%%%%%%%%%%%%%%%%%%%%%%%%%%%%%%%%%%%%%%%%%%%%%%%%%%%%%%%

Let us give schemes for construction functions for theories of type 2.
A clause is generated only if the conditions {\tt Neu(C,E)}, {\tt
Inv(C,I,E)}, etc. are satisfied. These conditions are not part of the
generated code.

{\small\begin{verbatim}
let f_C = function
| E, x when Neu(C,E) -> x
| x, E when Neu(C,E) -> x
| C(x,y), z when Assoc(C) -> f_C(x,f_C(y,z))
| x, y when Inv(C,I,E) -> insert_inv_C (f_I x) y
| x, y -> insert_C x y

and f_I = function
| E -> E
| I(x) -> x
| C(x,y) -> f_C(f_I y, f_I x)
| x -> I x

and insert_inv_C x y =
  try delete_C x y
  with Not_found -> insert_C (f_I x) y

and delete_C x = function
| Plus(y,_) when x < y -> raise Not_found
| Plus(y,t) when x = y -> t
| Plus(y,t) -> C(y, delete_C x t)
| y when y = x -> E
| _ -> raise Not_found



and insert_C x = function
| C(y,_) as u when x = y & idem -> u
| C(y,t) when x = y & nil -> f_C(A,t)
| C(y,_) as u when x <= y & com -> C(x,u)
| C(y,t) when Com(C) -> C(y, insert_C x t)
| u when x > u & Com(C) -> C(u,x)
| u when x = u & Idem(C) -> u
| u when x = u & Nil(C,A) -> A
| u -> C(x,u)
\end{verbatim}}

\begin{thm}
Let $\cE$ be the union of pairwise disjoint theories of type 1 or
2. Assume that, for all constructor $C$ which theory is of type $k$,
$f_C$ is defined as in Definition \ref{dfn-r} if $k=1$, and as above
if $k=2$. Then, $(f_C)_{C\in\cC}$ is valid wrt $\cE$.
\end{thm}

\begin{prf}
Assume that $\cE=\bigcup_{i=1}^n\cE_i$ where $\cE_1,\ldots,\cE_n$ are
pairwise disjoint theories of type 1 or 2. Whatever the type of
$\cE_i$ is, we saw that $\cE_i$ has a complete presentation
$\cR_i$. Therefore, since $\cE_1,\ldots,\cE_n$ share no symbol, by
definition of completion, the $AC$-completion of $\cE$ successfully
terminates with $\cR=\bigcup_{i=1}^n\cR_i$. Thus, $\a_{\cR,AC}$ is
terminating and $AC$-confluent. Since $\cF=(f_C)_{C\in\cC}$
computes $\cR,AC$-normal forms in $AC$-normal forms, by Theorem
\ref{thm-valid}, $\cF$ is valid.\cqfd\\
\end{prf}

The construction functions of type 2 can be easily extended to deal
with ring or lattice structures (distributivity and absorbance
equations).

More general results can be expected by using or extending results on
the modularity of completeness for the combination of rewrite
systems. The completeness of hierarchical combinations of
non-$AC$-rewrite systems is studied in \cite{rao93fsttcs}. Note
however that the modularity of confluence for $AC$-rewrite systems has
been formally established only recently in \cite{jouannaud06rta}.

Note that the construction function definitions of type 1 or 2 provide
the same results with call-by-value, call-by-name or lazy evaluation
strategy.

The detailed study of the complexity of theses definitions (compared
to AC-rewriting) is left for future work.
