%%%%%%%%%%%%%%%%%%%%%%%%%%%%%%%%%%%%%%%%%%%%%%%%%%%%%%%%%%%%%%%%%%%%%%%%%%%%%%
% ON THE EXISTENCE OF CONSTRUCTION FUNCTIONS
%%%%%%%%%%%%%%%%%%%%%%%%%%%%%%%%%%%%%%%%%%%%%%%%%%%%%%%%%%%%%%%%%%%%%%%%%%%%%%

\section{On the existence of construction functions}
\label{sec-ex}

In this section, we provide a general theorem for the existence of
valid families of construction functions based on rewriting theory. We
recall the notions of rewriting and completion. The interested reader
may find more details in \cite{dershowitz90book}.\vsp[1mm]

\noindent
{\bf Standard rewriting.} A {\em rewrite rule} is an ordered pair of
terms $(l,r)$ written $l\a r$. A rule is {\em left-linear} if no
variable occurs twice in its left hand side $l$.

As usual, the set $\pos(t)$ of {\em positions in $t$} is defined as a
set of words on positive integers. Given $p\in\pos(t)$, let $t|_p$ be
the subterm of $t$ at position $p$ and $t[u]_p$ be the term $t$ with
$t|_p$ replaced by $u$.

Given a finite set $\cR$ of rewrite rules, the {\em rewriting
relation} is defined as follows: $t\a_\cR u$ iff there are
$p\in\pos(t)$, $l\a r\in\cR$ and a substitution $\t$ such that
$t|_p=l\t$ and $u=t[r\t]_p$. A term $t$ is an {\em $\cR$-normal form}
if there is no $u$ such that $t\a_\cR u$. Let $=_\cR$ be the
symmetric, reflexive and transitive closure of $\a_\cR$.

A {\em reduction ordering} $\succ$ is a well-founded ordering (there
is no infinitely decreasing sequence $t_0\succ t_1\succ\ldots$) stable
by context ($C(..t..)\succ C(..u..)$ whenever $t\succ u$) and
substitution ($t\t\succ u\t$ whenever $t\succ u$). If $\cR$ is
included in a reduction ordering, then $\a_\cR$ is well-founded
(terminating, strongly normalizing).

We say that $\a_\cR$ is {\em confluent} if, for all terms $t,u,v$ such
that $u\al^*_\cR t\a^*_\cR v$, there exists a term $w$ such that
$u\a^*_\cR w\al^*_\cR v$. This means that the relation
$\al^*_\cR\a^*_\cR$ is included in the relation $\a^*_\cR\al^*_\cR$
(composition of relations is written by juxtaposition).

If $\a_\cR$ is confluent, then every term has at most one normal
form. If $\a_\cR$ is well-founded, then every term has at least one
normal form. Therefore, if $\a_\cR$ is confluent and terminating, then
every term has a unique normal form.\vsp[1mm]

%%%%%%%%%%%%%%%%%%%%%%%%%%%%%%%%%%%%%%%%%%%%%%%%%%%%%%%%%%%%%%%%%%%%%%%%%%%%%%
% standard completion
%%%%%%%%%%%%%%%%%%%%%%%%%%%%%%%%%%%%%%%%%%%%%%%%%%%%%%%%%%%%%%%%%%%%%%%%%%%%%%

\noindent
{\bf Standard completion.} Given a finite set $\cE$ of equations and a
reduction ordering $\succ$, the standard Knuth-Bendix completion
procedure \cite{bendix70book} tries to find a finite set $\cR$ of
rewrite rules such that:

\begin{lst}{\bu}
\item $\cR$ is included in $\succ$,
\item $\a_\cR$ is confluent,
\item $\cR$ and $\cE$ have same theory: ${=_\cE}={=_\cR}$.
\end{lst}

Note that completion may fail or not terminate but, in case of
successful termination, $\cR$-normalization provides a decision
procedure for $=_\cE$ since $t=_\cE u$ iff the $\cR$-normal forms of
$t$ and $u$ are syntactically equal.\vsp[2mm]

%%%%%%%%%%%%%%%%%%%%%%%%%%%%%%%%%%%%%%%%%%%%%%%%%%%%%%%%%%%%%%%%%%%%%%%%%%%%%%
% AC equations
%%%%%%%%%%%%%%%%%%%%%%%%%%%%%%%%%%%%%%%%%%%%%%%%%%%%%%%%%%%%%%%%%%%%%%%%%%%%%%

However, since permutation theories like commutativity or
associativity and commutativity together (written AC for short) are
included in no reduction ordering, dealing with them requires to
consider rewriting with pattern matching modulo these theories and
completion modulo these theories. In this paper, we restrict our
attention to AC.

\begin{dfn}[Associative-commutative equations]
Let $Com$ be the set of commutative constructors, \ie the set of
constructors $C$ such that $\cE$ contains an equation of the form
$C(x,y)=C(y,x)$. Then, let $\cE_{AC}$ be the subset of $\cE$ made of
the commutativity and associativity equations for the commutative
constructors, $=_{AC}$ be the smallest congruence relation containing
$\cE_{AC}$ and $\cE_{\neg AC}=\cE\moins\cE_{AC}$.
\comment{
\begin{center}
$\begin{array}{rl}
\cE_{AC}= & \{C(x,y)=C(y,x)\in\cE~|~C\in Com\}\\
& \cup~\{C(x,C(y,z))=C(C(x,y),z)\in\cE~|~C\in Com\}\\
\end{array}$
\end{center}
}
\end{dfn}

\noindent
{\bf Rewriting modulo AC.} Given a set $\cR$ of rewrite rules, {\em
rewriting with pattern matching modulo $AC$} is defined as follows:
$t\a_{\cR,AC} u$ iff there are $p\in\pos(t)$, $l\a r\in\cR$ and a
substitution $\t$ such that $t|_p=_{AC}l\t$ and $u=t[r\t]_p$. A
reduction ordering $\succ$ is {\em $AC$-compatible} if, for all terms
$t,t',u,u'$ such that $t=_{AC}t'$ and $ u=_{AC}u'$, $t'\succ u'$ iff
$t\succ u$. The relation $\a_{\cR,AC}$ is {\em confluent modulo $AC$}
if ${(\al^*_{\cR,AC}=_{AC}\a^*_{\cR,AC})}\sle
{(\a^*_{\cR,AC}=_{AC}\al^*_{\cR,AC})}$.\vsp[2mm]

%%%%%%%%%%%%%%%%%%%%%%%%%%%%%%%%%%%%%%%%%%%%%%%%%%%%%%%%%%%%%%%%%%%%%%%%%%%%%%
% AC-completion
%%%%%%%%%%%%%%%%%%%%%%%%%%%%%%%%%%%%%%%%%%%%%%%%%%%%%%%%%%%%%%%%%%%%%%%%%%%%%%

\noindent
{\bf Completion modulo AC.} Given a finite set $\cE$ of equations and
an $AC$-compatible reduction ordering $\succ$, completion modulo $AC$
\cite{peterson81jacm} tries to find a finite set $\cR$ of rules such
that:

\begin{lst}{\bu}
\item $\cR$ is included in $\succ$,
\item $\a_{\cR,AC}$ is confluent modulo $AC$,
\item $\cE$ and $\cR\cup\cE_{AC}$ have same theory:
${=_\cE}={=_{\cR\cup\cE_{AC}}}$.
\end{lst}

%%%%%%%%%%%%%%%%%%%%%%%%%%%%%%%%%%%%%%%%%%%%%%%%%%%%%%%%%%%%%%%%%%%%%%%%%%%%%%
% definition of complete presentations
%%%%%%%%%%%%%%%%%%%%%%%%%%%%%%%%%%%%%%%%%%%%%%%%%%%%%%%%%%%%%%%%%%%%%%%%%%%%%%

\begin{dfn}
A theory $\cE$ has a {\em complete presentation} if there is an
AC-com\-patible reduction ordering for which the $AC$-completion of
$\cE_{\neg AC}$ successfully terminates.
\end{dfn}

Many interesting systems have a complete presentation: (commutative)
mo\-no\-ids, (abelian) groups, rings, etc. See
\cite{hullot80thesis,lechenadec86book} for a catalog. Moreover,
there are automated tools implementing completion modulo AC. See for
instance \cite{cime2.02,gaillourdet03cade}.

%%%%%%%%%%%%%%%%%%%%%%%%%%%%%%%%%%%%%%%%%%%%%%%%%%%%%%%%%%%%%%%%%%%%%%%%%%%%%%
% AC-normal forms
%%%%%%%%%%%%%%%%%%%%%%%%%%%%%%%%%%%%%%%%%%%%%%%%%%%%%%%%%%%%%%%%%%%%%%%%%%%%%%

A term may have distinct $\cR,AC$-normal forms but, by confluence
modulo $AC$, all normal forms are $AC$-equivalent and one can easily
define a notion of normal form for $AC$-equivalent terms
\cite{hullot80thesis}:

\begin{dfn}[$AC$-normal form]
Given an associative and commutative constructor $C$, {\em
$C$-left-combs} (resp. {\em $C$-right-combs}) and their {\em leaves}
are inductively defined as follows:

\begin{lst}{--}
\item If $t$ is not headed by $C$, then $t$ is both a $C$-left-comb
and a $C$-right-comb. The {\em leaves} of $t$ is the one-element list
$\leaves(t)=[t]$.
\item If $t$ is not headed by $C$ and $u$ is a $C$-right-comb,
then $C(t,u)$ is a $C$-right-comb. The {\em leaves} of $C(t,u)$ is the
list $t::\leaves(u)$.
\item If $t$ is not headed by $C$ and $u$ is a $C$-left-comb,
then $C(u,t)$ is a $C$-left-comb. The {\em leaves} of $C(u,t)$ is the
list $\leaves(u)@[t]$, where $@$ is the concatenation.
\end{lst}

\noindent
Let $\orient$ be a function associating a kind of combs (left or
right) to every AC-constructor. Let $\le$ be a total ordering on
terms. Then, a term $t$ is in {\em $AC$-normal form wrt $\orient$ and
$\le$} if:

\begin{lst}{--}
\item Every subterm of $t$ headed by an AC-constructor $C$ is an
$\orient(C)$-comb whose leaves are in increasing order wrt $\le$.
\item For every subterm of $t$ of the form $C(u,v)$ with $C$
commutative but non-associative, we have $u\le v$.
\end{lst}
\end{dfn}

%%%%%%%%%%%%%%%%%%%%%%%%%%%%%%%%%%%%%%%%%%%%%%%%%%%%%%%%%%%%%%%%%%%%%%%%%%%%%%
% existence of AC-normal forms
%%%%%%%%%%%%%%%%%%%%%%%%%%%%%%%%%%%%%%%%%%%%%%%%%%%%%%%%%%%%%%%%%%%%%%%%%%%%%%

As it is well-known, one can put any term in $AC$-normal form:

\begin{thm}
\label{thm-ac}
Whatever the function $\orient$ and the ordering $\le$ are, every term
$t$ has an $AC$-normal form $t\!\ad_{AC}$ wrt $\orient$ and $\le$, and
$t=_{AC}t\!\ad_{AC}$.
\end{thm}

\begin{prf}
Let $\cA$ be the set of rules obtained by choosing an orientation for
the associativity equations of $\cE_{AC}$ according to $\orient$:

\begin{lst}{--}
\item If $\orient(C)$ is ``left'', then take $C(x,C(y,z))\a C(C(x,y),z)$.
\item If $\orient(C)$ is ``right'', then take $C(C(x,y),z)\a C(x,C(y,z))$.
\end{lst}

$\a_\cA$ is a confluent and terminating relation putting every subterm
headed by an AC-constructor into a comb form according to
$\orient$. Let $\comb$ be a function computing the $\cA$-normal form
of a term. Let now $\sort$ be a function permuting the leaves of combs
and the arguments of commutative but non-associative constructors to
put them in increasing order wrt $\le$. Then, the function
$\sort\circ\comb$ computes the $AC$-normal form of any term and
$\sort(\comb(t))=_{AC} t$.\cqfd\\
\end{prf}

%%%%%%%%%%%%%%%%%%%%%%%%%%%%%%%%%%%%%%%%%%%%%%%%%%%%%%%%%%%%%%%%%%%%%%%%%%%%%%
% existence of a valid family
%%%%%%%%%%%%%%%%%%%%%%%%%%%%%%%%%%%%%%%%%%%%%%%%%%%%%%%%%%%%%%%%%%%%%%%%%%%%%%

This naturally provides a decision procedure for $AC$-equivalence: the
function $\la xy.\sort(\comb(x))=\sort(\comb(y))$. It follows that
$\cR,AC$-normalization together with $AC$-normalization provides a
valid normalization function, hence the existence of a valid family of
construction functions:

\begin{thm}
\label{thm-ex}
If $\cE$ has a complete presentation, then there exists a valid family
of construction functions.
\end{thm}

\begin{prf}
Assume that $\cE$ has a complete presentation $\cR$. We define the
computation of normal forms as it is generally implemented in
rewriting tools. Let $\step$ be a function making an $\cR,AC$-rewrite
step if there is one, or failing if the term is in normal form. Let
$\norm$ be the function applying $\step$ until a normal form is
reached. Since $\cR$ is a complete presentation of $\cE$, by
definition of the completion procedure, $\sort\circ\comb\circ\norm$ is
a valid normalization function. Thus, by Theorem \ref{thm-g}, the
associated family of construction functions is valid.\cqfd\\
\end{prf}

The construction functions described in the proof are not very
efficient since they are based on rewriting with pattern matching
modulo AC, which is NP-complete \cite{benanav87jsc}, and do not take
advantage of the fact that, by definition of PDTs, they are only
applied to terms already in normal form. We can therefore wonder
whether they can be defined in a more efficient way for some common
equational theories like the ones of Figure \ref{fig-eq}.

%%%%%%%%%%%%%%%%%%%%%%%%%%%%%%%%%%%%%%%%%%%%%%%%%%%%%%%%%%%%%%%%%%%%%%%%%%%%%%
% some common equations
%%%%%%%%%%%%%%%%%%%%%%%%%%%%%%%%%%%%%%%%%%%%%%%%%%%%%%%%%%%%%%%%%%%%%%%%%%%%%%

\begin{figure*}
\centering\caption{Some common equations on binary constructors\label{fig-eq}}\vsp[2mm]
\begin{tabular}{|c|c|c|c|}\hline
\bf Name & \bf Abbrev & \bf Definition & \bf Example\\\hline
associativity & $Assoc(C)$ & $C(C(x,y),z)=C(x,C(y,z))$ & $(x+y)+z=x+(y+z)$\\\hline
commutativity & $Com(C)$ & $C(x,y)=C(y,x)$ & $x+y=y+x$\\\hline
neutrality & $Neu(C,E)$ & $C(x,E)=x$ & $x+0=x$\\\hline
inverse & $Inv(C,I,E)$ & $C(x,I(x))=E$ & $x+(-x)=0$\\\hline
idempotence & $Idem(C)$ & $C(x,x)=x$ & $x\et x=x$\\\hline
nilpotence & $Nil(C,A)$ & $C(x,x)=A$ & $x\oplus x=\bot$ (exclusive or)\\\hline
%absorbance & $Abs(C,A)$ & $C(x,A)=A$ & $x\times 0=0$\\\hline
%absorption & $Lat(C,D)$ & $C(x,D(x,y))=x$ & $x\ou(x\et y)=x$\\\hline
%distributivity & $Dist(C,D)$ & $C(D(x,y),z)=D(C(x,z),C(y,z))$ & $(x+y)\times z=(x\times z)+(y\times z)$\\\hline
\end{tabular}
\end{figure*}

%%%%%%%%%%%%%%%%%%%%%%%%%%%%%%%%%%%%%%%%%%%%%%%%%%%%%%%%%%%%%%%%%%%%%%%%%%%%%%
% checking the validity of a family
%%%%%%%%%%%%%%%%%%%%%%%%%%%%%%%%%%%%%%%%%%%%%%%%%%%%%%%%%%%%%%%%%%%%%%%%%%%%%%

Rewriting provides also a way to check the validity of construction
functions:

\begin{thm}
\label{thm-valid}
If $\cE$ has a complete presentation $\cR$ and $\cF=(f_C)_{C\in\cC}$
is a family such that, for all $C:\s_1..\s_n\pi\in\S$ and terms
$v_i\in Val(\s_i)$, $f_C(v_1..v_n)$ is an $\cR,AC$-normal form of
$C(v_1..v_n)$ in $AC$-normal form, then $\cF$ is valid.
\end{thm}

\begin{prf}
\begin{lst}{--}
\item Correctness. Let $C:\s_1..\s_n\pi\in\S$
and $v_i\in Val(\s_i)$. Since $f_C(v_1..v_n)$ is an $\cR,AC$-normal
form of $C(v_1..v_n)$, we clearly have $f_C(v_1..v_n)=_\cE
C(v_1..v_n)$.

\item Completeness. 
Let $C:\s_1..\s_n\pi\in\S$, $v_i\in Val_\cF(\s_i)$,
$D:\tau_1..\tau_p\pi\in\S$, and $w_i\in Val_\cF(\tau_i)$ such that
$C(v_1..v_n)=_\cE D(w_1..w_p)$. Since $\cR$ is a complete presentation
of $\cE$, $\norm(C(v_1..v_n))=_{AC}\norm(D(w_1..w_p))$. Thus,
$f_C(v_1..v_n)= f_D(w_1..w_p)$.\cqfd\\
\end{lst}
\end{prf}

It follows that rewriting provides a natural way to explain what are
the possible values of an RDT: values are $AC$-normal forms matching
no left hand side of a rule of $\cR$.
