\section{Introduction}

Many algorithms use data types with some additional invariants. Every
function creating a new value from old ones must be defined so that
the newly created value satisfy the invariants whenever the old ones
so do.

One way to easily maintain invariants is to use abstract data types
(ADT): the implementation of an ADT is hidden and construction and
observation functions are provided. A value of an ADT can only be
obtained by recursively using the construction functions. Hence, an
invariant can be ensured by using appropriate construction
functions. Unfortunately, abstract data types preclude pattern
matching, a very useful feature of modern programming languages
\cite{ocaml3.09,haskell03,moreau03cc,tom2.3}. There have been various
attempts to combine both features in some way.

In \cite{wadler87popl}, P. Wadler proposed the mechanisms of {\em
views}. A view on an ADT $\alpha$ is given by providing a concrete
data type (CDT) $\gamma$ and two functions $in:\alpha\a\g$ and
$out:\g\a\alpha$ such that $in\circ out=id_\g$ and $out\circ
in=id_\alpha$. Then, a function on $\alpha$ can be defined by matching
on $\g$ (by implicitly using $in$) and the values of type $\g$
obtained by matching can be injected back into $\alpha$ (by implicitly
using $out$). However, by leaving the applications of $in$ and $out$
implicit, we can easily get inconsistencies whenever $in$ and $out$
are not inverses of each other. Since it may be difficult to satisfy
this condition (consider for instance the translations between
cartesian and polar coordinates), these views have never been
implemented. Following the suggestion of W. Burton and R. Cameron to
use the $in$ function only \cite{burton93jfp}, some propositions have
been made for various programming languages but none has been
implemented yet \cite{burton96,okasaki98ml}.

In \cite{burton93jfp}, W. Burton and R. Cameron proposed another very
interesting idea which seems to have attracted very little
attention. An ADT must provide construction and observation
functions. When an ADT is implemented by a CDT, they propose to also
export the constructors of the CDT but only for using them as patterns
in pattern matching clauses. Hence, the constructors of the underlying
CDT can be used for pattern matching but not for building values: only
the construction functions can be used for that purpose. Therefore,
one can both ensure some invariants and offer pattern matching. These
types have been introduced in \ocaml\ by the third author
\cite{weis03} under the name of {\em concrete data type with private
constructors}, or {\em private data type} (PDT) for short.\vsp[2mm]

Now, many invariants on concrete data types can be related to some
equational theory. Take for instance the type of $list$ with the
constructors $[]$ and $::$. Given some elements $v_1..v_n$, the sorted
list which elements are $v_1..v_n$ is a particular representative of
the equivalence class of $v_1$::..::$v_n$::$[]$ modulo the equation
$x$::$y$::$l$=$y$::$x$::$l$. Requiring that, in addition, the list
does not contain the same element twice is a particular representative
modulo the equation $x$::$x$::$l$=$x$::$l$.

Consider now the type of join lists with the constructors $empty$,
$singleton$ and $append$, for which concatenation is of constant
complexity. Sorting corresponds to associativity and commutativity of
$append$. Requiring that no argument of $append$ is $empty$
corresponds to neutrality of $empty$ wrt $append$. We have a structure
of commutative monoid.

More generally, given some equational theory on a concrete data type,
one may wonder whether there exists a representative for each
equivalence class and, if so, whether a representative of $C(t_1\ldots
t_n)$ can be efficiently computed knowing that $t_1\ldots t_n$ are
themselves representatives.

In \cite{thompson86lfp,thompson90scp}, S. Thompson describes a
mechanism introduced in the Miranda functional programming language
for implementing such non-free concrete data types without precluding
pattern matching. The idea is to provide conditional rewrite rules,
called {\em laws}, that are implicitly applied as long as possible on
every newly created value. This can also be achieved by using a PDT
which construction functions (primed constructors in
\cite{thompson86lfp}) apply as long as possible each of the laws.
Then, S. Thompson studies how to prove the correctness of functions
defined by pattern matching on such {\em lawful types}. However, few
hints are given on how to check whether the laws indeed implement the
invariants one has in mind. For this reason and because reasoning on
lawful types is difficult, the law mechanism was removed from
Miranda.\vsp[2mm]

In this paper, we propose to specify the invariants by unoriented
equations (instead of rules). We will call such a type a {\em
relational data type} (RDT). Sections \ref{sec-pdt} and \ref{sec-rdt}
introduce private and relational data types. Then, we study when an
RDT can be implemented by a PDT, that is, when there exist
construction functions computing some representative for each
equivalence class. Section \ref{sec-ex} provides some general
existence theorem based on rewriting theory. But rewriting may be
inefficient. Section \ref{sec-genr} provides, for some common
equational theories, construction functions more efficient than the
ones based on rewriting. Section \ref{sec-moca} presents \moca, an
extension of \ocaml\ with relational data types whose construction
functions are automatically generated. Finally, Section
\ref{sec-futur} discusses some possible extensions.
