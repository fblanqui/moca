\documentclass{article}
\title{A Gentle Introduction to \moca}
\author{Richard Bonichon \footnote{{\sf richard.bonichon@loria.fr}}}
\date{March 2008}

\usepackage{amsmath}
\usepackage{latexsym}
\usepackage{multirow}
\usepackage{geometry}
\usepackage{xspace, alltt}
\usepackage{fullpage}
%\usepackage{hevea}
\usepackage{ifthen}
\usepackage[T1]{fontenc}
\usepackage[latin1]{inputenc}
\usepackage[a4paper,pdftex,pdfstartview=FitH]{hyperref}
\usepackage{amssymb}
\usepackage{color}
\usepackage{graphicx}
\usepackage[leftbars]{changebar}
\usepackage[english]{babel}
\usepackage{fancyhdr}
\usepackage{titlesec}
\usepackage{amsmath,amssymb,amsthm}
\theoremstyle{plain} 
\newtheorem{definition}{Definition}
\theoremstyle{remark}
\newtheorem{remark}{Note}

\def\bitem{\item[$\bullet$]}

\def\moca{{\sf Moca}\xspace}
\def\zenon{{\sf Zenon}\xspace}
\def\focal{{\sf Focal}\xspace}
\def\ocaml{{\sf OCaml}\xspace}
\def\coq{{\sf Coq}\xspace}
\def\tamed{{\sf TaMed}\xspace}
\def\why{{\sf Why}\xspace}
\def\zenonweb{{\sf http://focal.inria.fr/zenon}\xspace}
\def\mocaweb{{\sf http://moca.inria.fr/}\xspace}
\def\mocacvs{{\sf http://camlcvs.inria.fr/cgi-bin/cvsweb/bazar-ocaml/moca/}\xspace}
\def\vnumber{{\sf 0.5.0}}
\def\C{$C$\xspace}

\newcommand{\todo}[1]{\hspace*{-3cm}
  \begin{minipage}{1.0\linewidth}
    {\bf TODO: \\ #1}
  \end{minipage}
}

\newcommand{\note}[1]{{\bf NOTE:} #1}
\newcommand{\ift}[2]{If \C has #1 then
\[
  \begin{array}{ll}
   #2 
  \end{array}
\]
}

\newcommand{\chaptitle}[1]{
\part*{#1}
\setcounter{section}{0}
\addcontentsline{toc}{part}{\protect{#1}}
\def\chapname{#1}
}

\newcommand{\secpage}[1]{section \ref{#1}, p.\pageref{#1}}
\newcommand{\mysec}[1]{Sec.\ref{#1}\xspace}

\newcommand{\pagetitle}[2]{
\vspace*{3cm}
\thispagestyle{empty}
\begin{center}
  {\Huge\bf #1}
\end{center}

\vspace*{2cm}
\begin{center}
  {\Large #2}
\end{center}
\vspace*{1cm}
\begin{center}
  {\Large Richard Bonichon}
\end{center}
\vfill
\def\postes{#2}
}


\usepackage{listings}

% \lstdefinelanguage{Moca}[Objective]{Caml}
% {morekeywords={rule, associative, commutative, nilpotent, inverse,
%     opposite, neutral, idempotent, distributive, absorbing, absorbent,
%   left, right, involutive, status, precedence, completion},%
% }%

\lstloadlanguages{[Objective]Caml,Moca}

\definecolor{lstbg}{gray}{0.98} \definecolor{lstfg}{gray}{0.10}
\definecolor{lstrule}{gray}{0.6} \definecolor{lstnum}{gray}{0.4}
\definecolor{lsttxt}{rgb}{0.3,0.2,0.6}
\newcommand{\lstbrk}{\mbox{$\color{blue}\scriptstyle\cdots$}}

\def\lp@basic{
  \ifmmode\normalfont\mathtt\mdseries\scriptsize
  \else\normalfont\ttfamily\mdseries\scriptsize
  \fi
}

\def\lp@inline{
  \ifmmode\normalfont\mathtt\scriptstyle
  \else\normalfont\ttfamily\mdseries\small
  \fi}

\def\lp@keyword{}
\def\lp@special{\color{lstfg}}
\def\lp@comment{\normalfont\ttfamily\mdseries}
\def\lp@string{\color{lstfg}}
\def\lp@ident{}
\def\lp@number{\rmfamily\tiny\color{lstnum}}

\lstdefinestyle{moca-style}{%
  basicstyle={\ifmmode\normalfont\mathtt\scriptstyle
  \else\normalfont\ttfamily\mdseries\small
  \fi},%
  identifierstyle={},%
  commentstyle={\normalfont\ttfamily\mdseries},%
  keywordstyle={\ifmmode\mathsf\else\sffamily\fi},%
  keywordstyle=[2]\color{lstfg},%
  stringstyle={\color{lstfg}},%
  emphstyle={}\underbar,%
  showstringspaces=false,%
  mathescape=true,%
  numberstyle={\rmfamily\tiny\color{lstnum}},%
  xleftmargin=6ex,xrightmargin=2ex,%
  framexleftmargin=1ex,%
  frame=l,%
  framerule=1pt,%
  rulecolor=\color{lstrule},%
  backgroundcolor=\color{lstbg},%
  moredelim={*[s]{/*@}{*/}},%
  moredelim={*[l]{//@}},
  morecomment={[is]{//NOPP-BEGIN}{NOPP-END}},
  mathescape=true,
  escapechar=`
% breaklines is broken when using a inline and background
%  breaklines,prebreak={\lstbrk},postbreak={\lstbrk},breakindent=5ex %
}

\lstdefinestyle{moca}{
  language={[Objective]Caml},
  alsolanguage=Moca,
  style=moca-style
}

\lstset{
  language=Moca,
  style=moca,
  basicstyle={ \ifmmode\normalfont\mathtt\mdseries\scriptsize
  \else\normalfont\ttfamily\mdseries\scriptsize
  \fi},
  stepnumber=1,
  numbers=left
}

% \lstset{
%   language=Moca,
%   captionpos=b,
%   xleftmargin=10mm,
%   xrightmargin=10mm,
% %%  frame=lines,
%   columns=[c]{fixed},
% %%  numbers=right,
%   stepnumber=2,
%   numberfirstline=false
% }

\def\mlm{{\sf .mlm}\xspace}
\def\mlms{{\sf .mlms}\xspace}
\def\cons{{\sf cons}\xspace}
\def\insertf{{\sf insert}\xspace}
\def\delete{{\sf delete}\xspace}
\def\potent{{\sf potent}\xspace}
\def\return{{\sf return}\xspace}
\def\compare{{\sf compare}\xspace}
\def\mrule{{\sf rule}\xspace}
\def\testd{{\sf test}\xspace}
\def\na{{\it not applicable}.}
\def\rw{\longrightarrow}
\newcommand{\ret}[1]{{\sf return\_#1}\xspace}

\newenvironment{foo}[0]{~\begin{\itemize}}{\end{itemize}}


\begin{document}
  \begin{abstract}
   You will (hopefully!) find in this tutorial the complement of the
already available documentation for \moca. \moca relies heavily on the
\ocaml programming language: some prior knowledge of \ocaml is
therefore recommended but non-\ocaml functional
programmers\footnote{Should other programmers be called
  ``dysfunctional'' programmers?} should have no problem reading the examples. 

You can download stable versions from \mocaweb\ or browse the cvs
archive from \mocacvs.

\end{abstract}
\maketitle
\tableofcontents



\section{Introduction}
\label{sec:intro}

\subsection{What is \moca ?}
\moca\ is a general construction functions generator for \ocaml\ data types
with invariants.

\moca\ allows the high-level definition and automatic management of
complex invariants for data types. In addition, \moca\ provides the
automatic generation of maximally shared values, independantly or in
conjunction with the declared invariants.

A relational data type is a concrete data type that declares
invariants or relations that are verified by its constructors. For
each relational data type definition, \moca\ compiles a set of
construction functions that implements the declared relations.

\moca\ supports two kinds of relations:
\begin{itemize}
\item algebraic relations (such as associativity or commutativity of a
binary constructor), 
\item general rewrite rules that map some pattern constructors
of \moca\ and variables to some arbitrary user's define expression.
\end{itemize}

Algebraic relations are primitive, so that \moca\ ensures the
correctness of their treatment. By contrast, the general rewrite rules
are under the programmer's responsability, so that the desired
properties must be verified by a programmer's proof before compilation
(including for completeness, termination, and confluence of the
resulting term rewriting system).

Algebraic invariants are specified by using keywords denoting
equational theories like commutativity and associativity. \moca\
generates construction functions that allow each equivalence class to
be uniquely represented by their canonical value.

For theoretical details, have a look at \cite{moca07} which should be
included in the distribution.

\subsection{Building \moca}

First download the last stable version from \mocaweb. Or do a checkout
of the current cvs archive as follows\footnote{The line {\sf export
    CVSROOT=:pserver:anoncvs@camlcvs.inria.fr:/caml} works with {\sf
    bash} or {\sf zsh}. Read your shell manual if you do not know how
  to set environment variables with your shell.}:

\begin{verbatim}
export CVSROOT=:pserver:anoncvs@camlcvs.inria.fr:/caml
cvs login (* hit the enter key at the CVS password prompt*)
cvs checkout bazar-ocaml/moca
\end{verbatim}

Whatever you have done in the previous step, read {\em carefully} the
README and INSTALL files even if it will most probably be enough to
type the usual: 
\begin{verbatim}
./configure
make
make install
\end{verbatim}

The BSD-style manpage of \moca is intended to be a handy full documentation of
the software once you have mastered it enough. Do not hesitate to use
it as well!

\section{Features}
\label{sec:features}

As of version \vnumber\ the following features are available:

For the upcoming versions the development will focus on the following themes:
\begin{itemize}
\item Completion
\item Limited strategies
\end{itemize}


\section{(Lightweight) programming examples}
\label{sec:lipex}


\subsection{Sets}

Introduces keywords on a simple structure (is in {\sf test/set.mlm}).

\begin{lstlisting}[float, caption={Sets in \moca}, captionpos=b]
type 'a t = private
  | Empty
  | Singleton of 'a
  | Union of 'a t * 'a t
    begin
     associative
     commutative
     neutral (Empty)
     idempotent
    end
;;
\end{lstlisting}

\note{Be careful not to choose \ocaml keywords for your generators!
  This will be rejected by the \ocaml compiler. \moca does not check
  against such possible conflicts.} \\
\note{\moca generates identifiers as moca\_nameofid. Try not to
  capture them: they are as of now not ``keywords'' strictly speaking.}

\subsection{Ordered lists}
\label{sec:olist}

\begin{lstlisting}[caption={Ordered lists}]
  type 'a olist = private
  | Cons of  'a * 'a olist
     begin
      rule Cons (x, Cons (y, l)) when x < y  -> Cons (y, Cons (x, l))
     end
  | Nil
;;
\end{lstlisting}

\note{For any type t, the right-hand side of rules should be of type
  $\rightarow$ t in the, otherwise you'll get a typing error due to a
  mismatch between the types of your generated .mli and .ml. The only
  way to force another type on the right-hand side is as of now to
  regenerated an interface using ocamlc -i.
}
\section{More programming examples}
\label{sec:mpex}

There is already a small amount of examples in the {\sf examples}
directory of the stable distribution of \moca. There are even more in
the {\sf test} of the cvs directory. But beware! some of them are work
in progress and, as such, will maybe not compile.

\par We will take some of them to illustrate the main features of
\moca\ and also highlight some peculiarities. The code is mainly taken
from the {\sf test/proptab} and {\sf test/fotab} directories of the
cvs sources.

\note{No type inference}
\subsection{Propositional tableaux}
\label{sec:proptab}

The name ``tableaux'' is the general term given to a 
family of proof-search methods. Tableaux methods exist for a wide
range of logics, including the classical propositional case, which we
will implement using \moca. More about this subject can be found in
the classical book \cite{Smu68} of R. Smullyan or even on the web (see
for example {\sf http://en.wikipedia.org/wiki/Method\_of\_analytic\_tableaux}).

\subsubsection{Disjunctive Normal Form}
We consider the following pretty standard definition of booleans,
where atoms are represented by names (hence {\sf string}).

Notice that we do not use the names {\sf True, False, And} for our
constructors as
the generated construction functions would have been called 
{\sf true, false, and}, which are keywords of \ocaml.

\begin{lstlisting}
type boolean = private
  | Btrue | Bfalse
  | Batom of string
  | Bnot of boolean
  | Band of boolean * boolean
  | Bor of boolean * boolean
  | Bimplies of boolean * boolean 
;;
\end{lstlisting}

\subsubsection{The whole process}
See listing \ref{lst:proptab}
\begin{lstlisting}[float, caption={Propositional tableaux}, label={lst:proptab}]
 type boolean = private
  | Btrue | Bfalse
   
  | Batom of string
 
  | Bnot of boolean
     begin 
         involutive 
         distributive (Band, Bor) 
         distributive (Bor, Band) 
     end 

  | Band of boolean * boolean
        begin
           associative 
           idempotent 
           commutative 
           distributive (Bor) 
           inverse(Bnot, Bfalse)
        end


  | Bor of boolean * boolean
        begin
          associative  
(*           (\* The following means that if we have twice the same branch *)
(*              they will be nearby and only one instance will be kept *)
(*              in the bor comb *)
(*           *\) *)
           commutative 
          idempotent
          inverse (Bnot, Btrue)
          neutral (Bfalse)
          absorbent (Btrue)
        end

   | Bimplies of boolean * boolean 
         begin 
           rule Bimplies (x, y) -> Bor (Bnot x, y) 
         end
;;

\end{lstlisting}
\subsection{First-order tableaux}
\label{sec:fotab}

In this is an example, functions are declared in quotient type.
Take care of the evaluation order (for skolemization).


\section{Advanced features}
\label{sec:adv}

This section describes the use of more advanced \moca features, which
may be incomplete. 

\subsection{Testing normalization}
\label{sec:testing}

\moca has an automated feature to test your realtional types, which
you can activate using the following flag at your prompt:
\begin{verbatim}
$ moca -test file.mlm
\end{verbatim}

Tests are randomly generated, so if you happend to encounter a bug you
want to chase down, you can replay the same exact test set by using
the {\sf -seed} flag and giving it the seed number of your previously
generated test set.



\subsection{Completion}
\label{sec:comp}

\begin{lstlisting}{float, caption={Incomplete group}}
  type 'a t = private
  | Unit
      begin
        completion precedence 10
      end
  | Gen of 'a
      begin
        completion precedence 20
      end
  | Opp of 'a t
      begin
        completion precedence 40
      end
  | Mult of 'a t * 'a t
      begin
        completion precedence 30
        associative
        neutral left (Unit)
        neutral right (Unit)
        opposite (Opp)
      end
;;
\end{lstlisting}

\section{And now ?}
\label{sec:conc}

Here we are at the end of this tutorial. Hopefully you can now develop
your own programs using \moca and \ocaml.

We have not talked about some of the optional flags that can be
activated on the command-line. They are described in the man page. Try
them !

%\nocite{*}
\bibliographystyle{alpha}
\bibliography{biblio}

\end{document}
