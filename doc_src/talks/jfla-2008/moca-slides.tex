\newslide{Relational data types}

\advitransition{slide}

A {\em relational} data type (or RDT) is a private data type with declared {\em relations}.

The relations define the invariants that must be verified by the values of the
type.

The notion of relational data type is {\em not} native to the Caml compiler: it is
provided via an external program generator that generates regular Caml code for a
relational data type definition.

\newslide{The Moca framework}

\advitransition{slide}

{\em Moca} provides a notation to state predefined algebraic relations between
constructors,

Moca provides a notation to define arbitrary rewritting rules between
constructors.

Moca provides a module generator, \verb"mocac", that generates code to
implement a corresponding normal form.

Team: Fr�d�ric Blanqui \& Pierre Weis (Researchers), Richard Bonichon (Post
Doc), Laura Lowenthal (Internship), Th�r�se Hardin (Professor Lip6).

See \verb|http://moca.inria.fr/|.

\newslide{High level description of relations}
\advitransition{wipe}

We consider relational data types defined using:
\begin{citemize}
\item nullary or constant constructors,
\item unary or binary constructors,
\item nary constructors (argument has type $\alpha$ list).
\end{citemize}

Arguments cannot be {\em too complex} (in particular functionnal).

\newslide{Properties of constructors}
\advitransition{wipe}

A binary constructor $op$ of an RDT {\tt t} can be declared as:

\begin{citemize}
 \item 
    \verb"associative" meaning that
    $\forall x, y, z \in {\mbox {\tt t}}: (x\ op\ y)\ op\ z = x\ op\ (y\ op\ z)$,
 \item
    \verb"commutative" meaning that
    $\forall x, y \in {\mbox {\tt t}}: x\ op\ y = y\ op\ x$,
 \item
    \verb"distributive" with respect to another binary operator $opp$ in
{\tt t} meaning that
    $\forall x, y, z \in {\mbox {\tt t}}: (x\ opp\ y)\ op\ z = (x\ op\ y)\ opp\ (y\ op\ z)$,
\end{citemize}

\newslide{Properties of constructors}
\advitransition{wipe}

A binary constructor $op$ of a RDT {\tt t} can be declared as:

\begin{citemize}
 \item
    having $e$ as its \verb"neutral" meaning that
    $\forall x \in {\mbox {\tt t}}: x\ op\ e = e\ op\ x = x$,
 \item
    having $opp$ as \verb"opposite" meaning that
    $\exists e \in {\mbox {\tt t}}, e $ is neutral for $op$, and
    $\forall x \in {\mbox {\tt t}}: x\ op\ (opp\ x) = (opp\ x)\ op\ x = e $,
 \item
    having $z$ as its \verb"absorbent" element meaning that
    $\forall x \in {\mbox {\tt t}}: x\ op\ z = z\ op\ x = z$,
\end{citemize}

\newslide{Properties of constructors}
\advitransition{wipe}

A unary constructor $op$ of a RDT {\tt t} can be declared as:

\begin{citemize}
 \item
    being \verb"idempotent" meaning that
    $\forall x \in {\mbox {\tt t}}:\ op\ (op\ x) =\ op\ x$,
 \item
    being \verb"nilpotent" wrt $z$ meaning that
    $\forall x \in {\mbox {\tt t}}:\ op\ (op\ x) = z$,
 \item
    being \verb"involutive" meaning that
    $\forall x \in {\mbox {\tt t}}:\ op\ (op\ x) = x$,
\end{citemize}

\newslide{Defining arbitrary relations}
\advitransition{wipe}

A constructor $op$ of a RDT {\tt t} can have one or more
rewrite rules declared as:

\begin{citemize}
 \item
    \verb"rule" $op$ \verb"pat" $\rightarrow$ \verb"expr"
    meaning that any occurrence of pattern $op$ \verb"pat" has to be
rewritten as \verb"expr"
\end{citemize}

Example:
\begin{verbatim}
rule Bool_not (Bool_true) -> Bool_false
\end{verbatim}

\vfill

\newslide{The mocac compiler}

From these specifications, the {\em mocac} compiler generates the construction
functions that build the normal form of values that verifies the algebraic
relations and the invariants of a relational type.

The {\em mocac} compiler is a module generator for RDTs.

The input for mocac is a file with suffix \verb".mlm": it is a regular Caml
file with specific annotations to define the relations.

\newslide{Examples}

A trivial example with no annotations:

\begin{verbatim}
type bexpr = private
   | Band of bexpr list
   | Bor of bexpr list
   | Btrue
   | Bfalse;;
\end{verbatim}

\newslide{Generated files}

Interface:

\begin{verbatim}
type bexpr = private
   | Band of bexpr list
   | Bor of bexpr list
   | Btrue
   | Bfalse;;
val bfalse : bexpr
val band : bexpr list -> bexpr
val bor : bexpr list -> bexpr
val btrue : bexpr
\end{verbatim}

\newslide{Generated files}

Implementation:
\begin{verbatim}
type bexpr =
   | Band of bexpr list
   | Bor of bexpr list
   | Btrue
   | Bfalse

let rec bfalse = Bfalse
and band x = Band x
and bor x = Bor x
and btrue = Btrue
\end{verbatim}

\newslide{{\tt .mlm} source file}

A more realistic example for boolean expressions:
\begin{verbatim}
type bexpr = private
   | Band of bexpr * bexpr
   begin
     associative
     commutative
     distributive (Bxor)
     neutral (Btrue)
     absorbing (Bfalse)
     opposite (Binv)
   end
\end{verbatim}

\newslide{{\tt .mlm} source file}

\begin{verbatim}
  | Bxor of bexpr * bexpr
  begin
    associative
    commutative
    neutral (Bfalse)
    opposite (Bopp)
  end
\end{verbatim}

\newslide{{\tt .mlm} source file}

\begin{verbatim}
  | Btrue
  | Bfalse
  | Bvar of string

  | Bopp of bexpr
  begin
    rule Bopp(Btrue) -> Btrue
  end

  | Binv of bexpr;;
\end{verbatim}

\newslide{Generated interface}
\begin{verbatim}
type bexpr = private
  | Band of bexpr * bexpr
  (*
    associative
    commutative
    distributive (Bxor)
    neutral (Btrue)
    absorbing (Bfalse)
    opposite (Binv)
  *)
  ...
\end{verbatim}

\newslide{Generated implementation}

Type definition + simple operators
\begin{verbatim}
type bexpr = ...

let rec bvar x = Bvar x
and bopp x =
  match x with
  | Btrue -> Btrue
  | Bfalse -> Bfalse
  | Bopp x -> x
  | Bxor (x, y) -> bxor (bopp x, bopp y)
  | _ -> Bopp x
and bfalse = Bfalse
\end{verbatim}

\newslide{Generated implementation}

Binary associative + commutative operators are more tricky
\begin{verbatim}
and band z =
  match z with
  | Bfalse, _ -> Bfalse
  | _, Bfalse -> Bfalse
  | Btrue, y -> y
  | x, Btrue -> x
  | Binv x, y -> insert_opp_in_band x y
  | x, Binv y -> insert_opp_in_band y x
  | Bxor (x, y), z -> bxor (band (x, z), band (y, z))
  | x, Bxor (y, z) -> bxor (band (x, y), band (x, z))
  | Band (x, y), z -> band (x, band (y, z))
  | x, y -> insert_in_band x y
\end{verbatim}

\newslide{Generated implementation}

Insertion in a {\tt band} comb
\begin{verbatim}
and insert_in_band x u =
  match u with
  | Band (Binv y, t) when y = x -> t
  | Band (y, t) when x <= y ->
      begin try delete_in_band (Binv x) u with
        Not_found -> Band (x, u)
      end
  | Band (y, t) -> Band (y, insert_in_band x t)
  | Binv y when y = x -> Btrue
  | _ when x < u -> Band (x, u)
  | _ -> Band (u, x)
\end{verbatim}

\newslide{Generated implementation}

Deletion in a {\tt band} comb (note that {\tt band} is commutative)
\begin{verbatim}
and insert_opp_in_band x u =
  match u with
  | Band (y, t) when y = x -> t
  | Band (y, t) -> Band (y, insert_opp_in_band x t)
  | _ when x = u -> Btrue
  | _ -> insert_in_band (Binv x) u
and delete_in_band x u =
  match u with
  | Band (y, t) when y = x -> t
  | Band (y, (Band (_, _) as t)) -> Band (y, delete_in_band x t)
  | Band (y, t) when x = t -> y
  | _ -> raise Not_found
\end{verbatim}

\newslide{Generated implementation}

The inverse operator cannot be defined on the absorbing element...
\begin{verbatim}
and binv x =
  match x with
  | Bfalse -> failwith "Division by Absorbing element"
  | Btrue -> Btrue
  | Binv x -> x
  | Band (x, y) -> band (binv x, binv y)
  | _ -> Binv x
and btrue = Btrue
and bxor z = ...
\end{verbatim}

\newslide{{\tt .mlm} source file}

Two binary operators and their associated (ring-like) stuff:
\begin{verbatim}
type aexpr = private
  | Add of aexpr * aexpr
  begin
    associative
    commutative
    neutral (Zero)
    opposite (Opp)
  end
\end{verbatim}

\newslide{{\tt .mlm} source file}

\begin{verbatim}
  | Mul of aexpr * aexpr
  begin
    associative
    commutative
    distributive (Add)
    neutral (One)
    absorbing (Zero)
    opposite (Inv)
  end
  | One
  | Zero
  | Var of string
  | Opp of aexpr
  | Inv of aexpr;;
\end{verbatim}

\newslide{Generated interface}

Just regular: export the RDT type and its construction functions:
\begin{verbatim}
type aexpr = private
  | Add of aexpr * aexpr ...

val var : string -> aexpr
val opp : aexpr -> aexpr
val mul : aexpr * aexpr -> aexpr
val inv : aexpr -> aexpr
val add : aexpr * aexpr -> aexpr
val zero : aexpr
val one : aexpr
\end{verbatim}

\newslide{Generated implementation}
\begin{verbatim}
type aexpr =
   | Add of aexpr * aexpr
 ...

let rec var x = Var x
and opp x =
  match x with
  | Zero -> Zero
  | Opp x -> x
  | Add (x, y) -> add (opp x, opp y)
  | _ -> Opp x
\end{verbatim}

\newslide{Generated implementation}

Binary operators:
\begin{verbatim}
and mul z =
  match z with
  | Zero, _ -> Zero
  | _, Zero -> Zero
  | One, y -> y
  | x, One -> x
  | Inv x, y -> insert_opp_in_mul x y
  | x, Inv y -> insert_opp_in_mul y x
  | Add (x, y), z -> add (mul (x, z), mul (y, z))
  | x, Add (y, z) -> add (mul (x, y), mul (x, z))
  | Mul (x, y), z -> mul (x, mul (y, z))
  | x, y -> insert_in_mul x y
\end{verbatim}

\newslide{Generated implementation}

Insertion
\begin{verbatim}
and insert_in_mul x u =
  match u with
  | Mul (Inv y, t) when y = x -> t
  | Mul (y, t) when x <= y ->
      begin try delete_in_mul (Inv x) u with
      | Not_found -> Mul (x, u)
      end
  | Mul (y, t) -> Mul (y, insert_in_mul x t)
  | Inv y when y = x -> One
  | _ when x < u -> Mul (x, u)
  | _ -> Mul (u, x)
\end{verbatim}

\newslide{Generated implementation}

Deletion
\begin{verbatim}
and insert_opp_in_mul x u =
  match u with
  | Mul (y, t) when y = x -> t
  | Mul (y, t) -> Mul (y, insert_opp_in_mul x t)
  | _ when x = u -> One
  | _ -> insert_in_mul (Inv x) u
and delete_in_mul x u =
  match u with
  | Mul (y, t) when y = x -> t
  | Mul (y, (Mul (_, _) as t)) -> Mul (y, delete_in_mul x t)
  | Mul (y, t) when x = t -> y
  | _ -> raise Not_found
\end{verbatim}

\newslide{Generated implementation}

Definition of inverse, and so on
\begin{verbatim}
and inv x =
  match x with
  | Zero -> failwith "Division by Absorbing element"
  | One -> One
  | Inv x -> x
  | Mul (x, y) -> mul (inv x, inv y)
  | _ -> Inv x
...
and zero = Zero
and one = One
\end{verbatim}

\newslide{Maximal sharing generation}

The moca compiler also provides values represented as maximally shared trees.

You just have to use the \verb"-sharing" option of the compiler.

Hence the \verb".mlm" source file for maximally ``arith'' values is
the same.

\newslide{Generated interface}

The interface is slightly modified to incorporate the hash codes into values:
\begin{verbatim}
type info = { mutable hash : int };;
type aexpr = private
   | Add of info * aexpr * aexpr
   ...
;;
\end{verbatim}

\newslide{Generated interface}
Construction functions are similar; an additional equality function is
also provided (to benefit from the sharing to get fast equality with \verb"==")
\begin{verbatim}
val var : string -> aexpr
...
val eq_aexpr : aexpr -> aexpr -> bool
\end{verbatim}

\newslide{Generated implementation}

The implementation defines the types and the hash code generator:
\begin{verbatim}
type info = { mutable hash : int }
type aexpr =
   | Add of info * aexpr * aexpr
    ...

let mk_info h = {hash = h}
\end{verbatim}

\newslide{Generated implementation}

The implementation defines an equality to share values:
\begin{verbatim}
let rec equal_aexpr x y = x == y;;
\end{verbatim}

\newslide{Generated implementation}

Then the hash key access functions for the RDT
\begin{verbatim}
let rec get_hash_aexpr x =
  match x with
  | Add ({hash = h}, _x1, _x2) -> h
  | Mul ({hash = h}, _x1, _x2) -> h
  | Var ({hash = h}, _x1) -> h
  | Opp ({hash = h}, _x1) -> h
  | Inv ({hash = h}, _x1) -> h
  | One -> 1
  | Zero -> 0
\end{verbatim}

\newslide{Generated implementation}

Then the hash code computation function
\begin{verbatim}
let rec hash_aexpr x =
  succ
    (match x with
     | Add (_, x1, x2) ->
         get_hash_aexpr x1 + (get_hash_aexpr x2 + Obj.tag (Obj.repr x))
     | Mul (_, x1, x2) ->
         get_hash_aexpr x1 + (get_hash_aexpr x2 + Obj.tag (Obj.repr x))
     | Var (_, x1) -> Hashtbl.hash x1 + Obj.tag (Obj.repr x)
     | Opp (_, x1) -> get_hash_aexpr x1 + Obj.tag (Obj.repr x)
     | Inv (_, x1) -> get_hash_aexpr x1 + Obj.tag (Obj.repr x)
     | One -> 1
     | Zero -> 0)
\end{verbatim}

\newslide{Generated implementation}

Then those functions are encapsulated into a weak hash table:
\begin{verbatim}
module Hashed_aexpr =
  struct type t = aexpr let equal = equal_aexpr let hash = hash_aexpr end

module Shared_aexpr = Weak.Make (Hashed_aexpr)

let table_aexpr = Shared_aexpr.create 1009
\end{verbatim}

\newslide{Generated implementation}

The basic construction functions use sharing:
\begin{verbatim}
let rec mk_Add x1 x2 =

  let info = {hash = 0} in

  let v = Add (info, x1, x2) in

  let _ = info.hash <- hash_aexpr v in
  try Shared_aexpr.find table_aexpr v with
  | Not_found -> let _ = Shared_aexpr.add table_aexpr v in v
...
\end{verbatim}

\newslide{Generated implementation}

Then the normalisation functions also use the maximal sharing (calling
{\tt mk\_Add}, {\tt mk\_Opp}):
\begin{verbatim}
let rec var x = mk_Var x
and opp x =
  match x with
  | Zero -> Zero
  | Opp (_, x) -> x
  | Add (_, x, y) -> add (opp x, opp y)
  | _ -> mk_Opp x

and mul z = ...
and zero = Zero
and one = One
\end{verbatim}

\newslide{Current state of {\em mocac}}

We use a KB completion tool to complete the user's set of relations.

We generate automatic test beds for the generated construction functions.

We wrote a paper at ESOP'07: it states the framework, provides definitions of
the desired construction functions, proves the correctness of the
construction functions in simple cases.

\newslide{Future work}

Still need to:

\begin{citemize}
\item prove the generated code (i.e. provide a proof for each generated
implementation),
\item or prove the code generator (better: once and for all).
\end{citemize}

Not so easy \verb":("

We need also to integrate/interface {\em mocac} to other frameworks:
\begin{citemize}
\item for Focal (more work to do, need pattern matching first),
\item for Tom/Gom (Pierre-�tienne Moreau, INRIA Lorraine) ?
\end{citemize}
