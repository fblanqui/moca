\begin{frame}
\frametitle{La proc�dure g�n�rale}
\framesubtitle{Comment fait-on ?}

\texttt{
\begin{tabular}{l}
testi 5\\
(\alert<3>{let x =} \alert<4>{element 5} in\\
 \alert<3>{let y =} \alert<4>{concat (element 60, element 47)} in\\
 \alert<3>{let z =} \alert<4>{empty} in\\
 \alert<2>{concat (concat (\alert<3>{x}, \alert<3>{y}), \alert<3>{z}) =}\\
                  \hspace{0.35\textwidth} \alert<2>{concat (\alert<3>{x}, concat (\alert<3>{y}, \alert<3>{z})))}
\end{tabular}
}

\pause
\begin{exampleblock}{Plusieurs �l�ments � g�n�rer}
\begin{itemize}
\item<2-> Des �quations
\item<3-> Des substitutions
\item<4-> Des valeurs
\end{itemize}
\end{exampleblock}
\end{frame}
