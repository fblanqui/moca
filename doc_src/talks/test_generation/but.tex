\begin{frame}
\frametitle{Le but}
\framesubtitle{Mais qu'est-ce que tu entends par g�n�ration automatique?}

\begin{block}{Nouvelle option pour mocac}
\begin{tabular}{l c l}
\texttt{mocac -test exemple.mlm} & $\rightarrow$ &
     \texttt{exemple.mli}\\
  && \texttt{exemple.ml}\\
  && \alert{\texttt{exemple\_test.ml}}
\end{tabular}
\end{block}

\begin{block}{Entr�e}
\begin{itemize}
  \item Ensemble de d�clarations de types
    \begin{itemize}
    \item Juste les types \emph{variants} priv�s
    \item D�finitions des constructeurs, leur type, leurs relations
    \end{itemize}
  \item Nom du module, options, etc.
\end{itemize}
\end{block}

\begin{block}{Sortie}
Code de test, � ex�cuter avec le module \texttt{Gentest} de Moca
\end{block}
\end{frame}

\begin{frame}[fragile]
\frametitle{Exemple}
\framesubtitle{sequence.mlm}
\begin{verbatim}
(* The type of polymorphic sequences of elements.
   Repetition is allowed. Order is relevant. *)

type 'a t = private
   | Empty
   | Element of 'a
   | Concat of 'a t * 'a t
   begin
    associative
    neutral (Empty)
   end
;;
\end{verbatim}
\end{frame}

\begin{frame}[fragile]
\frametitle{Exemple}
\framesubtitle{sequence\_test.ml}
\begin{verbatim}
open Sequence;;
open Gentest;;

testing "Sequence (automatic)";;

testi 0 (let x = element 21 in
         concat (x, empty) = x)
...
testi 5
(let x = element 5 in
 let y = concat (element 60, element 47) in
 let z = empty in
 concat (concat (x, y), z) =
                  concat (x, concat (y, z)))
\end{verbatim}
\end{frame}
