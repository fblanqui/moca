\newslide{The new Moca release}

\advitransition{slide}

The {\em 0.6} release of the Moca compiler was delivered on February the
$13^{th}$ of 2008.

The main new feature is the {\em automatic test generation} for mocac
generated files. This has been done by Laura Lowenthal during her internship
at Loria (extremely successful internship achievement).

\newslide{The new Moca release}

\advitransition{slide}

In addition, we did:

\begin{citemize}
  \item Numerous bug fixes (in particular thanks to Laura' automatic tests).
  \item Completion has been enhanced.
  \item Distributivity has been widely generalized.
  \item Vary-ary generation completely revisited.
  \item Some bug additions (for instance {\tt Division\_by\_absorbent} removal).
\end{citemize}

\newslide{Embedded user's Caml code}

\advitransition{slide}

An important new feature is the possibility for the user to define
arbitrary Caml code within the definition of the relational type. This was
mandatory to define a specific comparison function when the regular Caml
polymorphic {\tt Pervasives.compare} is not semantically sound for the
relation type. The user's code is output as is, {\em after} the definition of the
relational type, but {\em before} the beginning of the definition of the
construction functions.

As usual, the documentation has been improved, in particular by the addition
of the ESOP article, the JFLA talk, and various other talks given about moca.

\newslide{The next Moca release(s)}

\advitransition{slide}

We will split the future development of {\em Moca} into two parts:

\begin{citemize}
  \item the implementation part,
  \item the research part.
\end{citemize}

The implementation part is shorter term and more practical. It is also well understood.

The research part is long term and may be impractical and/or unclear, half
backed, or not understood at all.

\newslide{The implementation part plan}

\advitransition{slide}

We split the implementation plan into:

\begin{citemize}
  \item revisit the algebraic keywords specification,
  \item enhance the internal test bed,
  \item enhance the automatic test generation procedure,
  \item write the manual for Moca.
\end{citemize}

\newslide{Implementation: specification}

\advitransition{slide}

For each algebraic keyword, we must precisely define its behaviors for each
of its variations. In particular, concerning arity of generators, we must fix
the vocabulary:

\begin{citemize}
  \item constant generator (or constary ?),
  \item unary generator,
  \item binary generator (two arguments),
  \item listary generator (a.k.a. vary-ary or vary-adic),
  \item multary generator (a.k.s. multi-ary i.e. any arity).
\end{citemize}

\newslide{Implementation: vocabulary}

\advitransition{slide}

By definition:
\begin{citemize}
  \item a {\em constary} generator has {\em no argument},
  \item a {\em unary} generator (has {\em one} argument which is {\em not a list}),
  \item a {\em binary} generator (has {\em two} arguments),
  \item a {\em listary} generator (has {\em one} argument which is {\em a list}),
  \item a {\em multary} generator (has any number of arguments).
\end{citemize}

\newslide{Implementation: specification}

\advitransition{slide}

For each keyword and each arity, give:

\begin{citemize}
  \item applicability to each arity, applicability w.r.t implied or required property,
  \item the rule(s) generated (match clause),
  \item the priority w.r.t. other rules (?),
  \item the ``no values of the relational type matches'' statement,
  \item specify the {\tt Left}, {\tt Right}, and {\tt Both} behavior.
\end{citemize}

No macro expansion in the parser.mly syntax !

\newslide{Implementation: the test directory}

\advitransition{slide}

\begin{citemize}
  \item in each case of the keyword specifications, add a test bed to check
  the behavior of the compiler w.r.t. the specification,
  \item enhance the internal test bed to check as much as possible
  the combination of algebraic rules,
  \item complete the set of test files to handle the usual mathematical
  structures (fields, vectorial spaces, etc.) and in particular usual
  generators/relations presentations of groups (as found in Coxeter and alii),
  \item enhance the automatic test generation procedure to handle polymorphic
  relational type and all the examples given in the test directory,
  \item move some test files to the examples for the moca's users.
\end{citemize}

\newslide{The research part}

\advitransition{slide}

We will split the future research development of {\em Moca} into:

\begin{citemize}
  \item the Test,
  \item the Completion,
  \item the Focalize Library,
  \item the Proofs,
  \item Moca for Focalize,
  \item Moca for the Caml programmer,
\end{citemize}

\newslide{Research: the Test generation}
\advitransition{wipe}

Try to understand the generality of the test generation procedure:
\begin{citemize}
\item how to generalize the procedure to user's defined equations (relations) ?
\item how to generalize the procedure to the full Caml language ?
\end{citemize}


\newslide{Research: Completion algorithm}
\advitransition{wipe}

\begin{citemize}
\item Generalize usage of automatic completion,
\item AC completion ?
\item How to generate complex rules via completion ?
      e.g. {\tt Absorbent} + {\tt Inverse} induce {\tt
      Division\_by\_absorbent};
      the generation function for {\tt Inverse} needs rule
      {\tt A -> failwith "Division by absorbent"},
      {\em UNLESS} $ A = E $ holds.
\end{citemize}

\newslide{Research: the Focalize library}
\advitransition{wipe}

\begin{citemize}
 \item Take the Focalize library and ``implement'' it using Moca's algebraic
    rules.
 \item Implement the associated algorithm of the Focalize library ?
\end{citemize}

\newslide{Research: Proofs}
\advitransition{wipe}

Write proofs, proofs, and proofs!!!

\begin{citemize}
 \item write by hand a proof of a simple example of moca generated code,
 \item understand how to generalize the preceding proof to generate a proof
 with the generated code (file {\tt file\_coq.v} ?)
\end{citemize}

Gather and carefully state the generic properties and proofs of the Moca
generated code, to be able to go on next point.

\newslide{Research: Moca for Focalize}
\advitransition{wipe}

Interface Moca to Focalize:

\begin{citemize}
 \item add a {\tt private} type facility to Focalize data type definitions,
 \item interface Moca to Focalize to allow relational data type definitions
 in Focalize,
 \item add the relevant lemmas and properties for the construction functions
 to the Focalize code (free proofs to generate for Zenon and Coq).
\end{citemize}

\newslide{Research: Moca for Caml}
\advitransition{wipe}

Use Moca to generate {\tt .mli} files that we do not want to write.

\begin{citemize}
  \item from a {\tt .mlm} file with additional annotations:
     \begin{citemize}
        \item {\tt export} clauses to export relevant identifiers,
        \item {\tt abstract} annotations for abstract data types,
        \item {\tt test} annotations to specify test equations or
        dis-equations ,
     \end{citemize}
  \item Other generations (similar to {\tt -test}) ?
     \begin{citemize}
        \item {\tt set} clause to generate set universes,
        \item {\tt data base} annotation to generate data bases,
        \item {\tt make\_string} annotation to generate a {\tt make\_string} (or
     printing function),
     \end{citemize}
\end{citemize}

\newslide{Development guidelines}

Program with in mind ``No confusion can ever arise'':

\begin{citemize}
 \item except for the value of some quantities, unknown at compile time,
 \item since it would be caught by the Caml compiler (exhaustive and fragile matches).
\end{citemize}
