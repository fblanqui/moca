\def\adviheader{\relax}

\newslide{The idea}
\advitransition{slide}

Enhance Caml data type definitions in order to

\begin{citemize}
\item handle {\em invariants} verified by values of a type,

\item provide quotient data types, in the sense of mathematical quotient
  structures,

\item define automatic computation of canonical representant of values.
\end{citemize}

\newslide{Usual data type definition kinds}
\advitransition{slide}

There are three classical {\em kinds} of data type definitions:
\begin{citemize}
\item {\em sum} type definitions
 (disjoint union of sets with tagged summands),

\item {\em product} type definitions
 (anonymous cartesian products)
 (cartesian products with named components)

\item {\em abbreviation} type definitions
 (short hands to name type expressions)
\end{citemize}

\newslide{Visibility of data type definitions}

There are two classical {\em visibility} of a data type definitions:

\begin{citemize}

\item {\em concrete} visibility: the implementation of the type is visible,

\item {\em abstract} visibility: the implementation of the type is hidden,
\end{citemize}

\newslide{Consequence of visibility for programmers}

For concrete types:

\begin{citemize}
\item value inspection is allowed via pattern matching,
\item value construction is not restricited,
\end{citemize}

For abstract types:

\begin{citemize}
\item value inspection is not possible,
\item value construction is carefully ruled.
\end{citemize}

\newslide{Consequence of visibility for programs}

For concrete types, the representation of values is manifest:

\begin{citemize}
\item the compiler can perform type based optimization,
\item the debugger (and the toplevel) can show (print) values.
\end{citemize}

For abstract types, the representation of values is hidden:
\begin{citemize}
\item the compiler cannot perform type based optimization,
\item the debugger and the toplevel system just print values as \verb"<abstr>".
\end{citemize}

\newslide{Visibility management constructs}

Modules are used to define visibility of data type definitions.

\begin{citemize}
  \item the implementation defines the data type as concrete,
  \item the interface exports the data type as concrete/or abstract.
\end{citemize}

The interface exports the data type as concrete if it {\em declares} the data
type with its definition (the associated constructors for a sum type, the labels for
a record, or the defining type expression for an abbreviation).

\newslide{Defining invariants}

Usual (concrete) data types implement {\em free} data structures:
\begin{citemize}
  \item sums: free (closed) algebra (the constructors define the
signature of the free algebra),
  \item products: free cartesian products for records,
  \item abbreviations: free type expressions.
\end{citemize}

By {\em free} we mean the usual mathematical meaning: no restriction on the
construction of values of the set (type), provided the signature constraints
are fulfilled.

\newslide{Examples}

\begin{alltt}
type expression =
   | Int of int
   | Add of expression * expression
   | Opp of expression

type id = \{
  firstname : string;
  lastname : string;
  married : bool;
\};;

type real = float;;
\end{alltt}

\newslide{Counter examples}

Sum and products:

\begin{alltt}
type positive_int = Positive of int;;

type rat = \{ numerator : int; denominator : int; \};;
\end{alltt}

Despite the intended meaning:
\begin{citemize}
\item \verb"Positive (-1)" is a valid \verb"positive_int" value,
\item \verb"{numerator = 1; denominator = 0;}" is a valid \verb"rat".
\end{citemize}

\newslide{Counter examples}

Abbreviations:

\begin{alltt}
type km = float;;
type mile = float;;
\end{alltt}
Despite the intended meaning:
\begin{citemize}
\item \verb"-1.0" is a valid \verb"km" value,
\item \verb"((x : km) : mile)" is not an error (a \verb"km" value is
a \verb"mile" value).
\end{citemize}

\newslide{Non free data types}

Many mathematical structures are not free.\\
(Cf. Generators \& relations presentations of mathematical structures.)

Many data structures are not free having various validity constraints.

The usual feature of programming languages to deal with non free data
structure is to provide abstract visibility and abstract data types (or ADT).

\newslide{ADT as Non free data type}

Using an ADT, the constructors, labels, or type expression synonym of the type are
no more accessible to build spurious undesired values.\\
{\em Construction} of values is restricted to {\em construction
functions} defined in the implementation module of the abstract data type.

Advantage: non free data types invariants are properly handled.\\
Drawback: inspection of values is no more a built in feature. Inspection
functions should be provided explicitely by the implementation module.

There is no pattern matching facility for ADTs.

\newslide{Example}

\begin{alltt}
type positive_int = Positive of int;;
let make_positive_int i =
  if i < 0 then failwith "negative int" else Positive i;;
let int_of_positive_int p = p;;

type rat = \{ numerator : int; denominator : int; \};;
let make_rat n d =
  if d = 0 then failwith "null denominator" else
  \{ numerator = n; denominator = d; \};;

let numerator r = r.numerator;;
let denominator r = r.denominator;;
\end{alltt}

\newslide{Example}

\begin{alltt}
type km = float;;
let make_km k =
  if k <= 0.0 then failwith "negative distance" else k;;

let float_of_km k = k;;

type mile = float;;
let make_mile m =
  if m <= 0.0 then failwith "negative distance" else m;;
let float_of_mile m = m;;
\end{alltt}

\newslide{Private visibility}
\advitransition{slide}

To provide pattern matching for non free data types, we introduced a new
visibility for data type definitions: the {\em private} visibility.

As a concrete data type, a private data type ({\em PDT}) has a manifest implementation.
As an abstract data type, a private data type limits the construction of
values to provided construction functions.

In short, private data type are:
\begin{citemize}
\item concrete data types that support {\em invariants} or {\em relations}
between their values,
\item fully compatible with pattern matching.
\end{citemize}

\newslide{Examples}
\advitransition{slide}

All the quotient sets you need can be implemented as private types.

For quotient types the corresponding invariant is:\\
any element in the private type is the canonical representant of its
equivalence class.

Formulas, groups, \ldots

\newslide{Definition of private data types}
\advitransition{wipe}

As abstract and concrete data types, private data types are implemented using modules:

- inside {\em implementation} of their defining module, relational data types
are regular concrete data types,

- in the {\em interface} of their defining module, private data types are simply
declared as {\em private}.

\newslide{Usage of a private data type}

In client modules:\\
\begin{citemize}
  \item a private data type does not provide labels nor constructors to build its values,
  \item a private data type provides labels or constructors for pattern matching.
\end{citemize}

\newslide{Consequences}
\advitransition{wipe}

The module that implements a private data type:

\begin{itemize}
 \item must export {\em construction functions} to build the values,
 \item has not to provide {\em destruction functions} to access inside
the values.
\end{itemize}

The pattern matching facility is available for private data types.

\vfill

\newslide{Comparison with abstract data types}

Abstract data types also provide invariants, but:

\begin{itemize}

 \item once defined, an ADT is {\em closed}: new functions on the ADT
are mere compositions of those provided by the module.

 \item once defined, a private data type is {\em still open}: arbitrary new
functions can be defined via pattern matching on the representation of
values.

\end{itemize}

\newslide{Consequences}

\begin{citemize}
  \item the implementation of an ADT is big (it basically includes the set
of functions available for the type),

  \item the implementation of a PDT is small (it only includes the set of functions
that provides the invariants),
  \item proofs can be simpler for PDT (we must only prove that the
mandatory construction functions indeed enforce the invariants).
\end{citemize}

\newslide{Consequences}

Clients of an ADT have to use the construction and destruction
functions provided with the ADT.

Clients of a PDT must use the construction functions, to preserve
invariants but pattern matching is still freely available.

All the functions defined on an PDT {\em respect} the PDT's invariants
(granted for free by the type-checker!)

